\documentclass[a4paper, technote, compsoc]{IEEEtran}
%\documentclass[a4paper]{article}

\usepackage[T1]{fontenc}    % use 8-bit T1 fonts
\usepackage[utf8]{inputenc} % allow utf-8 input
\usepackage{amsmath}
\usepackage{amssymb}
\usepackage{booktabs}
\usepackage{float}  % used to fix location of images i.e.\begin{figure}[H]
\usepackage{graphicx}  %needed to include png, eps figures
\usepackage{mwe}
\usepackage{url} % correct bad hyphenation here
\usepackage{xcolor}
\usepackage{xr}
\usepackage{subfiles}
\usepackage{physics}
\usepackage[hidelinks]{hyperref}

\externaldocument[M-]{main}


\usepackage[backend=biber,style=ieee]{biblatex} 
\bibliography{bibliography.bib} %your file created using JabRef

% https://tex.stackexchange.com/questions/246/when-should-i-use-input-vs-include
% \newcommand{\N}{\mathbb{N}}
\newcommand{\Z}{\mathbb{Z}}
\newcommand{\Q}{\mathbb{Q}}

\newcommand{\R}{\mathbb{R}}
\newcommand{\RR}{\mathbb{R}^2}
\newcommand{\RRR}{\mathbb{R}^3}

\renewcommand{\Pmu}{\mathcal{P}}
\renewcommand{\P}{\mathbb{P}}

\newcommand{\delt}{\Delta t}
\newcommand{\utk}{{\widetilde u}^k}
\newcommand{\ut}[1]{{\widetilde u}^{#1}}
\newcommand{\rhog}{\text{\boldmath{$\rho$}}}
\renewcommand{\eps}{\varepsilon}
\DeclareMathOperator{\sspn}{span}
\newcommand{\spn}[1]{\sspn\left(#1\right)}



\begin{document}

\onecolumn

% paper title
\title{Reduced Basis Method \\ With Time-Deforming Domains \\ \large{MWE: Linear Heat Equation}}

% author names 
\author{Enrique Millán Valbuena \\ \normalsize{463 426 8}}% <-this % stops a space
        
% The report headers
\markboth{M. Sc. Aerospace Engineering, TU Delft}%do not delete next lines
{Shell \MakeLowercase{\textit{et al.}}: Bare Demo of IEEEtran.cls for IEEE Journals}

% make the title area
\maketitle

\begin{abstract}
   The research objective is to build a Reduced Order Model (ROM) for a parametrized one-dimensional heat equation with a deforming boundary.

   The main goal is to have a simple problem to define and tackle all the implementation details of the reduction procedure, 
   to later on solve an actual problem with industrial or clinical application. 

   Both the main body of the PDE, the geometrical definition of the boundary and the boundary conditions will be parametrized.
   
   % A concise description of the reducing procedure is provided, together with a priori convergence rates for basis size estimation and a posteriori error estimators to certify the use of the Reduced Order Model.
   % Numerical examples to showcase computational costs and implementation details are designed, implemented and validated with the Manufactured Solutions Method. 
\end{abstract}

\begin{IEEEkeywords}
    Reduced Order Model, Moving Domain, FEM, DEIM, POD, Galerkin-Projection
\end{IEEEkeywords}

%\setcounter{tocdepth}{3}
\setcounter{tocdepth}{2}
\tableofcontents

\twocolumn

\newpage

\subfile{research_project/heat_linear_1d_fom.tex}
\newpage
\subfile{research_project/heat_linear_1d_rom.tex}
\newpage
\subfile{research_project/manufactured_solutions.tex}


\end{document}