\documentclass[a4paper, technote, compsoc]{IEEEtran}
%\documentclass[a4paper]{article}

\usepackage[T1]{fontenc}    % use 8-bit T1 fonts
\usepackage[utf8]{inputenc} % allow utf-8 input
\usepackage{amsmath}
\usepackage{amssymb}
\usepackage{booktabs}
\usepackage{float}  % used to fix location of images i.e.\begin{figure}[H]
\usepackage{graphicx}  %needed to include png, eps figures
\usepackage{mwe}
\usepackage{url} % correct bad hyphenation here
\usepackage{xcolor}
\usepackage{xr}
\usepackage{subfiles}

\externaldocument[M-]{main}

\usepackage[backend=biber,style=ieee]{biblatex} 
\bibliography{bibliography.bib} %your file created using JabRef


\begin{document}

% paper title
\title{Reduced Order Models \\ With Moving Domains}

% author names 
\author{Enrique Millán Valbuena \\ \normalsize{463 426 8}}% <-this % stops a space
        
% The report headers
\markboth{M. Sc. Aerospace Engineering, TU Delft}%do not delete next lines
{Shell \MakeLowercase{\textit{et al.}}: Bare Demo of IEEEtran.cls for IEEE Journals}

% make the title area
\maketitle

\begin{abstract}
   The research objective is to build a Reduced Order Model (ROM) for a two-dimensional parametrized unsteady PDE with a moving boundary:
   \begin{itemize}
      \item Heat diffusion problem.
      \item (Bonus) Graetz convection-diffusion problem.
   \end{itemize}
   Both the main body of the PDE and the geometrical definition of the moving boundary will be parametrized.

   A concise description of the reducing procedure is provided, together with a priori convergence rates for basis size estimation and a posteriori error estimators to certify the use of the Reduced Order Model.
   Numerical examples to showcase computational costs and implementation details are designed, implemented and validated with the Manufactured Solutions Method. 
\end{abstract}

\begin{IEEEkeywords}
    Reduced Order Model, Moving Domain, FEM, DEIM, POD, Galerkin-Projection
\end{IEEEkeywords}
    

\section*{Literature Review}
\subfile{literature_review/literature.tex}

\newpage
\section*{Research Structure}
\subfile{research_project/research.tex}

\end{document}