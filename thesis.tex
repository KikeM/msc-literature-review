% \documentclass[a4paper, technote, draft, compsoc]{IEEEtran}
\documentclass[a4paper, technote,compsoc]{IEEEtran}
%\documentclass[a4paper]{article}

\usepackage[toc,page,titletoc]{appendix}
\usepackage[T1]{fontenc}    % use 8-bit T1 fonts
\usepackage[utf8]{inputenc} % allow utf-8 input
\usepackage{amsmath}
% \usepackage{hyperref}
\usepackage{amssymb}
\usepackage{booktabs}
\usepackage{float}  % used to fix location of images i.e.\begin{figure}[H]
\usepackage{graphicx}  %needed to include png, eps figures
\usepackage{mwe}
\usepackage{url} % correct bad hyphenation here
\usepackage[dvipsnames]{xcolor}
% \usepackage{todonotes}
\usepackage{xr}
\usepackage{subfiles}
\usepackage{physics}
\usepackage{epigraph}
% \usepackage{savetrees}

\usepackage{todonotes,tocloft,xpatch,hyperref}
\hypersetup{
    colorlinks,
    linkcolor={red!50!black},
    citecolor={blue!50!black},
    urlcolor={blue!80!black}
}

% This is based on classicthesis chapter definition
\let\oldsec=\section
\renewcommand*{\section}{\secdef{\Sec}{\SecS}}
\newcommand\SecS[1]{\oldsec*{#1}}%
\newcommand\Sec[2][]{\oldsec[\texorpdfstring{#1}{#1}]{#2}}%

% https://tex.stackexchange.com/a/61267/11984
\makeatletter
\xapptocmd{\Sec}{\addtocontents{tdo}{\protect\todoline{\thesection}{#1}{}}}{}{}
\newcommand{\todoline}[1]{\@ifnextchar\Endoftdo{}{\@todoline{#1}}}
\newcommand{\@todoline}[3]{%
  \@ifnextchar\todoline
    {}
    {\contentsline{section}{\numberline{#1}#2}{#3}{}{}}%
}
\let\l@todo\l@subsection
\newcommand{\Endoftdo}{}

\AtEndDocument{\addtocontents{tdo}{\string\Endoftdo}}
\makeatother


\externaldocument[M-]{main}

\usepackage[style=ieee]{biblatex}
\bibliography{xampl}
% \usepackage[backend=biber,style=ieee]{biblatex} 
\bibliography{bibliography.bib} %your file created using JabRef

% https://tex.stackexchange.com/questions/246/when-should-i-use-input-vs-include
% \newcommand{\N}{\mathbb{N}}
\newcommand{\Z}{\mathbb{Z}}
\newcommand{\Q}{\mathbb{Q}}

\newcommand{\R}{\mathbb{R}}
\newcommand{\RR}{\mathbb{R}^2}
\newcommand{\RRR}{\mathbb{R}^3}

\renewcommand{\Pmu}{\mathcal{P}}
\renewcommand{\P}{\mathbb{P}}

\newcommand{\delt}{\Delta t}
\newcommand{\utk}{{\widetilde u}^k}
\newcommand{\ut}[1]{{\widetilde u}^{#1}}
\newcommand{\rhog}{\text{\boldmath{$\rho$}}}
\renewcommand{\eps}{\varepsilon}

\DeclareMathOperator{\sspn}{span}
\newcommand{\spn}[1]{\sspn\left(#1\right)}
\newcommand{\mytodo}[1]{\todo[inline, color=green!20, inlinewidth=\columnwidth]{#1}}

\begin{document}

\onecolumn

% paper title
\title{Skipping The Jacobian \\[5mm] \large{(Hyper) Reduced Order Models For Moving Meshes}}

% author names 
\author{Enrique Millán Valbuena \\ \normalsize{463 426 8}}% <-this % stops a space
        
% The report headers
\markboth{M. Sc. Aerospace Engineering, TU Delft}%do not delete next lines
{Shell \MakeLowercase{\textit{et al.}}: Bare Demo of IEEEtran.cls for IEEE Journals}

% make the title area
\maketitle

\begin{abstract}
   We present a Reduced Order Model (ROM) for a one-dimensional nonlinear gas dynamics problem:
   the isentropic piston.
   The main body of the PDE, 
   the geometrical definition of the moving boundary, 
   and the boundary conditions are parametrized.
   The full order model is obtained with a Galerkin finite element discretization,
   under the Arbitrary Lagrangian Eulerian formulation (ALE).
   To stabilize the system, an artificial viscosity term is included.
   The Reduced Basis to express the solution is obtained with the classical POD technique.
   To overcome the explicit use of the jacobian transformation, 
   typical in the context of moving domains,
   a system approximation technique is used.
   The (Matrix) Discrete Empirical Interpolation Method, (M)DEIM, allows us
   to work with a weak form defined in the physical domain (and hence the physical weak formulation)
   whilst maintaining an
   efficient assembly for the algebraic operators, 
   despite their change with every timestep.
   All in all, our approach is purely algebraic
   and the reduced model makes no use of full order structures, 
   thus achieving a perfect \textit{offline-online} split.
   A concise description of the reducing procedure is provided, 
   together with a posteriori error estimations, obtained via model truncation,
   to certify the Reduced Order Model.
   % Numerical examples to showcase computational costs and implementation details are designed, 
   % implemented and validated with the Manufactured Solutions Method.
\end{abstract}

\begin{IEEEkeywords}
    \centering
    Finite Elements, Galerkin, Reduced Order Models, 
    Moving Piston, Deforming Mesh, ALE, 
    (M)DEIM, POD, 
    Model Truncation
\end{IEEEkeywords}

\setcounter{tocdepth}{2}
\tableofcontents

\newpage
\listoftodos

\subfile{foreword.tex}
\newpage
\subfile{executive.tex}

\newpage
\twocolumn

\subfile{introduction.tex}
% \section*{Literature Review}
% \subfile{literature_review/literature.tex}

% \newpage
% \section*{Research Structure}
% \subfile{research_project/research.tex}
% \subfile{research_project/graph_layout.tex}
\newpage
\subfile{research_project/piston/burgers_1d_fom.tex}
\newpage
\subfile{research_project/piston/burgers_1d_rom.tex}
\newpage
\subfile{research_project/piston/burgers_1d_calibration.tex}
\newpage
\subfile{research_project/piston/burgers_1d_hrom.tex}
\newpage
\subfile{conclusions.tex}

\newpage
\printbibliography

\begin{appendices}
    \newpage
    \subfile{research_project/piston/piston_movement.tex}
    \newpage
    \clearpage
    \subfile{research_project/piston/svd_analysis.tex}
    \newpage
    \clearpage
    \subfile{research_project/piston/parameter_range_selection.tex}
\end{appendices}


\end{document}