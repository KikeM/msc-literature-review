\documentclass[a4paper, technote, compsoc, draft]{IEEEtran}
%\documentclass[a4paper]{article}


\usepackage[T1]{fontenc}    % use 8-bit T1 fonts
\usepackage[utf8]{inputenc} % allow utf-8 input
\usepackage{amsmath}
\usepackage{amssymb}
\usepackage{booktabs}
\usepackage{float}  % used to fix location of images i.e.\begin{figure}[H]
\usepackage{graphicx}  %needed to include png, eps figures
\usepackage{mwe}
\usepackage{url} % correct bad hyphenation here
\usepackage[dvipsnames]{xcolor}
\usepackage{todonotes}
\usepackage{xr}
\usepackage{subfiles}
\usepackage{physics}
\usepackage{epigraph}

\externaldocument[M-]{main}

\usepackage[backend=biber,style=ieee]{biblatex} 
\bibliography{bibliography.bib} %your file created using JabRef

% https://tex.stackexchange.com/questions/246/when-should-i-use-input-vs-include
% \newcommand{\N}{\mathbb{N}}
\newcommand{\Z}{\mathbb{Z}}
\newcommand{\Q}{\mathbb{Q}}

\newcommand{\R}{\mathbb{R}}
\newcommand{\RR}{\mathbb{R}^2}
\newcommand{\RRR}{\mathbb{R}^3}

\renewcommand{\Pmu}{\mathcal{P}}
\renewcommand{\P}{\mathbb{P}}

\newcommand{\delt}{\Delta t}
\newcommand{\utk}{{\widetilde u}^k}
\newcommand{\ut}[1]{{\widetilde u}^{#1}}
\newcommand{\rhog}{\text{\boldmath{$\rho$}}}
\renewcommand{\eps}{\varepsilon}

\DeclareMathOperator{\sspn}{span}
\newcommand{\spn}[1]{\sspn\left(#1\right)}
\newcommand{\mytodo}[1]{\todo[inline, color=green!20, inlinewidth=\columnwidth]{#1}}

\begin{document}

\onecolumn

% paper title
\title{Skipping The Jacobian \\[5mm] \large{Hyperreduced Order Models For Moving Domains}}

% author names 
\author{Enrique Millán Valbuena \\ \normalsize{463 426 8}}% <-this % stops a space
        
% The report headers
\markboth{M. Sc. Aerospace Engineering, TU Delft}%do not delete next lines
{Shell \MakeLowercase{\textit{et al.}}: Bare Demo of IEEEtran.cls for IEEE Journals}

% make the title area
\maketitle

\begin{abstract}
   We present a Reduced Order Model (ROM) for a one-dimensional gas dynamics problem:
   the moving piston.
   The main body of the PDE, 
   the geometrical definition of the moving boundary, 
   and the boundary conditions themselves are parametrized.
   The Full Order Model is obtained with a Galerkin semi-implicit Finite Element discretization.
   The Reduced Basis to express the solution is obtained with the classical POD technique.
   To overcome the explicit use of the jacobian transformation, 
   typical in the context of moving domains,
   a system approximation technique is used.
   The (Matrix) Discrete Empirical Interpolation Method, (M)DEIM, allows us
   to work with a weak form defined in the physical domain whilst maintaining an
   efficient assembly for the algebraic operators, 
   despite their evolution with every timestep.
   All in all, our approach to the construction of the Reduced Order Model is purely algebraic and makes no use of Full Order structures, 
   achieving a perfect \textit{offline-online} split.
   A concise description of the reducing procedure is provided, together with a posteriori error estimations to certify the Reduced Order Model.
   Numerical examples to showcase computational costs and implementation details are designed, implemented and validated with the Manufactured Solutions Method.
\end{abstract}

\begin{IEEEkeywords}
    Reduced Order Model, Gas Dynamics, Moving Piston, Deforming Mesh, Galerkin, FEM, DEIM, MDEIM, POD
\end{IEEEkeywords}

\setcounter{tocdepth}{2}
\tableofcontents

\newpage
\listoftodos

\subfile{foreword.tex}
\newpage
\subfile{executive.tex}

\newpage
\twocolumn

\subfile{introduction.tex}
% \section*{Literature Review}
% \subfile{literature_review/literature.tex}

% \newpage
% \section*{Research Structure}
% \subfile{research_project/research.tex}
% \subfile{research_project/graph_layout.tex}
\newpage
\subfile{research_project/piston/burgers_1d_fom.tex}
\newpage
\subfile{research_project/piston/burgers_1d_rom.tex}
\newpage
\subfile{research_project/piston/burgers_1d_results.tex}

\appendix
\newpage
\subfile{research_project/piston/piston_movement.tex}

\end{document}