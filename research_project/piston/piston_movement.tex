% \documentclass[a4paper, technote, compsoc]{IEEEtran}
\documentclass[../../thesis.tex]{subfiles}

\begin{document}

\section{Determination of Piston Movement Law}
\label{sec:appendix_piston_movement_law}
We impose the movement of the piston, 
so we need to make sure we do so in a physically meaningful way.
This appendix is born after making the mistake 
of believing that any sinusoidal function would do the job.
Since our flow will depart from rest, 
we need to set the piston motion 
so that at the initial instant the piston also
starts its movement from rest.

To derive the piston movement law, 
we depart from a force defined by the elemental harmonic functions,
\begin{subequations}
\begin{align}
        A \cos(\omega t) + B \sin(\omega t) &= m \ddot{x}(t), 
        \\
        x(0) &= 0, 
        \\
        \dot{x}(0) &= 0.
\end{align}
\end{subequations}
Integrating in time, we get
\begin{equation}
    \frac{A}{m \omega} \sin(\omega t) - \frac{B}{m \omega} \cos(\omega t) + C_1 = \dot{x}(t).
\end{equation}
If we enforce the initial condition of rest to find the value of the integration constant, 
we arrive to an incongruency,
\begin{equation}
    - \frac{B}{m \omega} + C_1 = 0 \rightarrow C_1 = \frac{B}{m \omega}.
\end{equation}
If we were to integrate in time again, 
due to the presence of the constant $C_1$, 
a linear term proportional to $\sim t$ would show up.

This is not the physical result we expected, 
given the fact that we departed from two linear harmonic functions
(which introduce and remove energy with fixed frequency and amount from the system). 
Hence a harmonic movement was expected, not one changing linearly in time.

This conflictive result comes from 
the sinusoidal term in the definition of the force moving the piston.
If we set $B=0$, the first integration constant will become identically zero, $C_1=0$. 
When this is the case, the linear term vanishes, and we recover a harmonic piston movement,
\begin{equation}
    \frac{A}{m \omega^2} \cos(\omega t) + C_2 = x(t).
\end{equation}
By setting the initial piston location, we get the value for $C_2$,
\begin{equation}
    x(t) = L_0 - \frac{A}{m \omega^2} \left(1 - \cos(\omega t)\right).
\end{equation}
If we now define $A$ such that $\frac{A}{m \omega^2}$ represents a fraction of the initial piston length, $\delta L_0$, 
we get a compact expression for the piston movement,
\begin{equation}
    x(t) = L_0\left[1 - \delta \left(1 - \cos(\omega t)\right)\right].
\end{equation}

\end{document}