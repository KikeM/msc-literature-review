% \documentclass[a4paper, technote, compsoc]{IEEEtran}
\documentclass[../../thesis.tex]{subfiles}

\newcommand{\inner}[2]{\left<#1, #2\right>}
\newcommand{\alemap}{\ensuremath{\mathcal{A}}}
\newcommand{\dt}{\ensuremath{\Delta t}}
\newcommand{\pexp}{\ensuremath{\frac{2\gamma}{\left(\gamma-1\right)}}}
\newcommand{\aleX}{\ensuremath{\mathcal{X}}}
\newcommand{\Ah}[1]{\ensuremath{\vb{#1}^{n+1}_h}}
\newcommand{\Ahn}[1]{\ensuremath{\vb{#1}^{n}_h}}



\begin{document}

\section{One-Dimensional Gas Dynamics}
\label{sec:1d_fom_heat_equation_full_order_model}
The Full Order Model for the parametrized one-dimensional piston problem is derived.

We depart from the compressible isentropic Navier Stokes equations to end up with a non-linear Burgers-like equation.
Therefore, this problem contains all the necessary ingredients to show how the ROM behaves 
in the presence of a non-linear term within a moving domain.

The model will be derived in the continuous, semi-discrete and fully discrete contexts for a generic parametrization and forcing term.
We shall use the Galerkin projection principle to find a weak form, which we later discretize using the Finite Element Method. 

We define the vector $\vec{\mu} \in \mathcal{P}$ to collect all the parameters present in the formulation.
Parameters will be present in the PDE's body, in the boundary conditions, or in the geometrical definition of the domain. 

We deal with our problem within a deforming domain in time, whose movement is known and not part of the solution:
\begin{equation*}
    \Omega(t, \vec{\mu}) := \left\{x \in \mathbb{R} : x \in \left[0, L(t, \vec{\mu})\right]\right\}.
\end{equation*}
From now on, we drop the dependency on time and the parameters unless it is strictly necessary. 

If the variational or Finite Element problem is not correctly stated in formal terms yet, we apologize in advance.
These kind of language formalities take time to settle and this is still a draft. 
The reader will be able to fill in any notational or definition gaps for the moment. 

\subsection{Physical Derivation}
The piston movement $L(t)$ is a real-valued smooth sinusoidal function,
\begin{equation}
    L(t) = L_0 \left[1 - \delta \left(1 - \cos \left(\omega t\right)\right)\right],
\end{equation}
where $\omega$ is the frequency at which it oscillates and $\delta \ll L_0$ is a scale variable to adjust how much it is displaced from its original position.
The length $L_0$ is defined to keep physical dimensions sound, but it will remain fixed to $L_0=1$ for the remaining of the problem.

To derive the equations of motion of the fluid inside the piston, 
we depart from the conservation of mass, momentum and the isentropic relation between pressure and density,
\begin{subequations}
\begin{align}
    \pdv{\rho}{t} + u \pdv{\rho}{x} + \rho \pdv{u}{x} &= 0, \\ 
    \pdv{u}{t} + u \pdv{u}{x} + \frac{1}{\rho}\pdv{p}{x} &= 0, \\
    p &= k \rho^{\gamma} \label{eq:isentropic_relation}. 
\end{align}
\end{subequations}
Body forces and viscosity have been neglected for the sake of simplicity\footnote{Viscosity terms will be introduced later on for numerical stability.}.
A difficulty we find in this system for the piston application is the determination of the boundary condition for density at the piston location.
Ideally, we only want to solve an equation for the velocity, where boundary conditions are easy to set.

To do so, we take the following steps:
\begin{enumerate}
    \item Remove the pressure gradient by means of the isentropic relation. 
    \item Relate explicitly velocity~$u$ to density~$\rho$. %with a necessary compatibility condition.
    \item Collect all the above into one equation in terms of velocity~$u$.
\end{enumerate}

\subsubsection*{Removal of the pressure gradient}
To remove the pressure gradient, we start by taking derivatives in the isentropic relation (\ref{eq:isentropic_relation}):
\begin{equation}
    \pdv{p}{x} = k \gamma \rho^{\gamma-1} \pdv{\rho}{x}
\end{equation}
Then, we recognize that the coefficients $k \gamma \rho^{\gamma-1}$ multiplying the spatial derivative of density are in fact the squared speed of sound.
From thermodynamics, the speed of sound~$a$ squared is the derivative of pressure with respect to density at constant entropy,
\begin{equation}
    a^2 = \left.\pdv{p}{\rho}\right|_S = k \gamma \rho^{\gamma-1}.
\end{equation}
Hence, the expression for the pressure gradients becomes
\begin{equation}
    \pdv{p}{x} = a^2 \pdv{\rho}{x},
\end{equation}
which can be plugged directly into the momentum equation. 
The system becomes
\begin{subequations}
\begin{align}
        \pdv{\rho}{t} + u \pdv{\rho}{x} + \rho \pdv{u}{x} &= 0, \\ 
        \pdv{u}{t} + u \pdv{u}{x} + \frac{a^2}{\rho}\pdv{\rho}{x} &= 0, \\
        a &= \sqrt{k \gamma} \rho^{\frac{\gamma-1}{2}}. 
\end{align}
\end{subequations}

\subsubsection*{Compatibility Condition between $u$ and $\rho$}
Our next step is to find a compatibility condition between the mass and momentum equations.
We define $u:=\mathtt{V}(\rho)$, which, by application of the chain rule, leads to the following equalities between the derivatives of $u$ and $\rho$ :
\begin{subequations}
    \begin{align}
        \pdv{u}{t} &= \mathtt{V}' \pdv{\rho}{t}, \\
        \pdv{u}{x} &= \mathtt{V}' \pdv{\rho}{x},
    \end{align}
\end{subequations}
where $\mathtt{V}'$ represents differentiation with respect to density.
Introducing these relations into the mass and the momentum equations to remove $u$, we obtain
\begin{subequations}
    \begin{align}
            \pdv{\rho}{t} + \pdv{\rho}{x} \left(\rho \mathtt{V}' + \mathtt{V}\right) &= 0, \\ 
            \mathtt{V}' \pdv{\rho}{t} + \pdv{\rho}{x} \left(\mathtt{V} \mathtt{V}' + \frac{a^2}{\rho}\right) &= 0.
    \end{align}
\end{subequations}
Since we have two equations for two unknowns, the system can only have a solution if the determinant is zero.
\mytodo{Why does the determinant need to be zero for the compatibility condition?}
\begin{equation}
    \begin{vmatrix}
        1 & \mathtt{V} + \rho \mathtt{V}'\\ 
        \mathtt{V}' &  \mathtt{V}\mathtt{V}' + \frac{a^2}{\rho} 
    \end{vmatrix}
    = 0
\end{equation}
This determinant implies that
\begin{equation}
    \mathtt{V}' \left(\mathtt{V} + \frac{a^2}{\rho \mathtt{V}'} - \mathtt{V} - \rho \mathtt{V}'\right) = 0.
\end{equation}
The above equation has two possible solutions, either 
\begin{equation}
    \mathtt{V}'=0 \rightarrow \mathtt{V} = C,
\end{equation} 
which is the constant solution, valid but uninteresting, or 
\begin{equation}
    \frac{a^2}{\rho \mathtt{V}'} - \rho \mathtt{V}' = 0 \rightarrow \mathtt{V}' = \pm \frac{a}{\rho}.
\end{equation}
The $\pm$ signs come up because there are two travelling waves, one to each side of the domain.
Since our piston is located at the right boundary of the domain, 
we choose the $-$ sign so that waves propagate left, and continue our derivation.
We can integrate~$\mathtt{V}'$ with respect to~$\rho$ to obtain a relation between the flow velocity~$u$ and the speed of sound~$a$:
\begin{equation}
    \mathtt{V} = - \int_{\rho_0}^{\rho} \frac{a}{\rho} d\rho 
\end{equation}
where $\rho_0$ is some reference density.
At this point it is convenient to use this reference density~$\rho_0$ to remove the constant~$k$ from the expression of the speed of sound:
\begin{subequations}
    \begin{align}
        a &= a(\rho) 
        \rightarrow 
        a_0 = a(\rho_0),
        \\
        a &= \sqrt{k \gamma} \rho^{\frac{\gamma-1}{2}} 
        \rightarrow
        a = a_0 \left(\frac{\rho}{\rho_0}\right)^{\frac{\gamma-1}{2}}.
    \end{align}    
\end{subequations}
Plugging this expression into the integral, we get
\begin{subequations}
    \begin{align}
        u = \mathtt{V} &= \frac{2a_0}{\gamma-1}\left(1 - \frac{a}{a_0}\right),
        \\[2mm]
        a &= a_0 - \frac{\gamma-1}{2}u. \label{eq:speed_of_sound_with_velocity}
    \end{align}
\end{subequations}

\subsubsection{Burgers-like Equation}
With all the above, we are now ready to obtain \textit{one} equation which contains the three departing ones.
In the momentum equation we first substitute the space derivative of density,
\begin{subequations}
    \begin{align}
        \pdv{u}{x} = \mathtt{V}' \pdv{\rho}{x} \rightarrow \pdv{\rho}{x} &= - \frac{\rho}{a} \pdv{u}{x},
        \\[2mm]
        \pdv{u}{t} + \left(u-a\right)\pdv{u}{x} &= 0.
    \end{align}    
\end{subequations}
Then, we express the speed of sound in terms of velocity (\ref{eq:speed_of_sound_with_velocity}),
which leads to a PDE containing a Burgers-like nonlinear term and 
forced convection driven by the static speed of sound, 
\begin{equation}
    \pdv{u}{t} + \frac{\gamma+1}{2}u \pdv{u}{x} - a_0\pdv{u}{x} = 0.
\end{equation}
This PDE, together with its boundary conditions and the boundary's prescribed movement, represents the mathematical model of interest for our work. 

\subsubsection{Complete Determination of Flow Variables}
Although we now have \textit{one} equation which accounts for mass conservation and the isentropic relation between pressure and density, we still have three variables. 
Computing density, pressure and the speed of sound in space and time still remains useful: 
verification of mass conservation, computation of the force exerted by the fluid at the piston, or any other secondary derivations.

Since the speed of sound~$a$ is defined as a function of~$u$ in Equation (\ref{eq:speed_of_sound_with_velocity}),
once the latter is given, we can obtain density~$\rho$ and pressure~$p$ as a function of flow velocity~$u$:
\begin{subequations}
    \begin{align}
        \rho &= \rho_0 \left(\frac{a}{a_0}\right)^{\frac{2}{\gamma-1}},
        \\
        p    &= p_0 \left(\frac{\rho}{\rho_0}\right)^{\gamma},
        \\[2mm]
        \left(\frac{a}{a_0}\right) &= 1 - \frac{\gamma-1}{2}\left(\frac{u}{a_0}\right), 
        \\[2mm]
        \left(\frac{\rho}{\rho_0}\right) &= \left(1 - \frac{\gamma-1}{2}\left(\frac{u}{a_0}\right)\right)^{\frac{2}{\gamma-1}},
        \label{eq:density_as_a_function_of_speed}
        \\[2mm]
        \left(\frac{p}{p_0}\right) &= \left(1 - \frac{\gamma-1}{2}\left(\frac{u}{a_0}\right)\right)^{\frac{2\gamma}{\left(\gamma-1\right)}}.
    \end{align}
\end{subequations}

% -----------------------------------------------------------------------------
\subsubsection{Mass Conservation}
Before we present the complete problem and its numerical solution, we comment 
on the integral equation for mass conservation.
This relation will not be used for the solution of the system, but it is a 
healthy check to perform after the calculation, to make sure we have 
correctly integrated problem.
Mass conservation is the least of requirements we demand from fluid motion.

For a control volume whose boundary moves with the piston, 
the integral expression for mass conservation is
\begin{equation}
    \frac{d}{dt}\int_{\Omega_t} \rho d\Omega 
    + \int_{\partial\Omega_t} \rho \left(\vec{u} - \vec{u}_c\right) \cdot \vec{n} dS = 0
\end{equation}
At the piston location the flow moves with the piston, $\vec{u} = \vec{u}_c$, 
so the boundary integral vanishes. 
There is no flux through the walls either, so the only contribution left is the outflow.
At this location, the scalar product of the velocity and normal vectors is 
negative\footnote{Although we expect the flow to leave the domain at this point, 
which means a negative magnitude, this products needs to be done taking 
the variables positive in the direction of the axes.}, which implies
\begin{equation}
    \frac{d}{dt}\int_{0}^{L(t)} \rho(x,t) dx - \rho(0,t) u(0,t) = 0.
    \label{eq:integral_mass_conservation}
\end{equation}
If we introduce the expression of density in terms of velocity from 
Equation~(\ref{eq:density_as_a_function_of_speed}), we get
\begin{equation}
\begin{split}
        \frac{d}{dt}\int_{0}^{L(t)} \left(1 - \frac{\gamma-1}{2}\left(\frac{u(x,t)}{a_0}\right)\right)^{\frac{2}{\gamma-1}} dx 
        \\
        - u(0,t) \left(1 - \frac{\gamma-1}{2}\left(\frac{u(0,t)}{a_0}\right)\right)^{\frac{2}{\gamma-1}}  = 0.
\end{split}
\end{equation}
To ease our task at verifying this equation, we rather calculate the integral term for each time step
\begin{equation}
    I(t) = \int_{0}^{L(t)} \left(1 - \frac{\gamma-1}{2}\left(\frac{u(x,t)}{a_0}\right)\right)^{\frac{2}{\gamma-1}} dx,
\end{equation}
and only then compute time derivatives, which can be done with a second order finite difference scheme.
% \begin{equation}
%     \frac{d}{dt}I(t) := I'(t) \simeq \frac{I\left(t^{n+1}\right) - I\left(t^{n}\right)}{\dt}.
% \end{equation}
Then we can compute numerically the mass defect $MD(t)$ between the time derivative of volume variation
and mass outflow at the open boundary,
\begin{equation}
    MD(t) = I'(t) - u(0,t) \left(1 - \frac{\gamma-1}{2}\left(\frac{u(0,t)}{a_0}\right)\right)^{\frac{2}{\gamma-1}}.
    \label{eq:mass_conservation_check}
\end{equation}
We shall verify to which degree of accuracy this relation is honoured by our numerical scheme.

\mytodo{Formulate mass conservation as a FOM functional to be computed with the ROM basis.}

\subsubsection{Interesting Physical Quantities}
In the craft of building reduced order models, it is certainly interesting to measure and study the error between the full and reduced solutions. 
However, in practical applications, one is often more concerned about aggregated magnitudes derived from raw flow variables, such as maximum flow, pressure gradients across channels, etc. 
A comparison between the full order and reduced order aggregated magnitudes will shed light on the quality and reproducibility of the reduction scheme.

We consider two magnitudes of interest:
\begin{enumerate}
    \item The outflow at the boundary.
    \item The pressure gradient across the channel.
\end{enumerate}
The outflow at the boundary $F_L(t)$ is given by
\begin{subequations}
    \begin{align}
        F_L(t) &= u(0,t) \rho(0,t),
        \label{eq:outflow_at_the_boundary}
        \\  
        F_L(t) &= u(0,t) \rho_0 \left(1 - \frac{\gamma-1}{2}\left(\frac{u(x,t)}{a_0}\right)\right)^{\frac{2}{\gamma-1}}
    \end{align}
\end{subequations}

The pressure gradient across the channel is the difference between pressure at the piston and at the outflow boundary,
\begin{subequations}
    \begin{align}
        \Delta_L p(t) &= - \frac{p(0,t) - p(L(t),t)}{L(t)},
        \label{eq:channel_pressure_gradient} 
        \\ 
        \Delta_L p(t) &= 
        - \frac{p_0}{L(t)} 
        \left[
            \left(1 - \frac{\gamma-1}{2}\left(\frac{u(0,t)}{a_0}\right)\right)^{\pexp}
        \right.
        \\ 
        & - 
        \left.
            \left(1 - \frac{\gamma-1}{2}\left(\frac{u(L,t)}{a_0}\right)\right)^{\pexp}
        \right]
        . \nonumber
    \end{align}
\end{subequations}
We have defined it so that when the piston is moving forward, pushing the gas out of the chamber, the piston pressure gradient is defined positive.

\subsection{Continuous Formulation}
We continue with the definition of the problem in the continuous setting: 
a differential model given by a PDE and its boundary condition, referred to as {strong formulation}; 
and its weak formulation derived with the Galerkin principle.

We introduce the ALE formulation, to account for the movement of the mesh.
Then, we will introduce a scaling to work with non-dimensional variables.
This will complete the strong formulation.

Then, once we have obtained a weak formulation via the Galerkin projection principle,
we will introduce a Dirichlet lifting of the boundary conditions.
To solve the FOM this is not mandatory, since the boundary conditions can be 
directly tweaked into the right-hand side of the algebraic vectors.
However, it is a paramount step for the construction of our reduced space,
to ensure the basis serves all boundary parametrizations.

\subsubsection{Strong Formulation}
At this point of the derivation, we allow ourselves to introduce a diffusion term which was not present in the physical derivation of the model.
The viscosity value will be small enough so that it disappears with the mesh size,

\mytodo{Review diffusion coefficient value for mesh consistency.}
\begin{equation}
    \varepsilon \sim \left(\Delta x\right)^2. 
\end{equation}

Thus, the differential model in the physical space is given by the following PDE, boundary and initial conditions,
\begin{subequations}
    \begin{align}
        \left.\frac{\partial u}{\partial t}\right|_{x} 
        + \frac{\partial u}{\partial x}\left(\frac{\gamma+1}{2}u - a_0\right) 
        - \varepsilon \pdv[2]{u}{x} &= 0, \label{eq:1d_fom_strong_pde} \\
        % \frac{\partial u}{\partial x}(0, t) &= 0, \\
        u(L, t) &= L'(t). 
    \end{align}
\end{subequations}
At the left boundary, we will impose a natural ouflow condition.
At the right boundary, the moving piston, a Dirichlet condition sets the flow velocity equal to the one of the moving wall,

\mytodo{Natural outflow condition means evaluating the weak form at the boundary.
This is compatible with the small diffusion coefficient.
This is equivalent to imposing an homogeneous Neumann boundary condition.
}
\begin{equation}
    L'(t) = - \delta L_0 \omega \sin(\omega t).
\end{equation}
The notation $\left.\frac{\partial u}{\partial t}\right|_{x}$ indicates that the derivative takes place in the physical moving domain. 
This is relevant for the ALE formulation, which we explain in the following section. 

\subsubsection{ALE Formulation}
Despite the fact that we will be solving the problem in the physical moving domain, 
we still need the two basic ingredients stemming at the root of the ALE method:
\begin{itemize}
    \item A smooth mapping between domains.
    \item A mesh velocity vector.
\end{itemize}

We introduce the ALE mapping $\alemap$ that connects a point in the fixed reference domain $\mathcal{X}$ with a point in the physical domain $x$:
\begin{subequations}
    \begin{align}
        x &= \alemap(\aleX, t), \\
        \aleX &= \alemap^{-1}(x, t),
    \end{align}
\end{subequations}
which we assume to be regular enough.

We define the mesh velocity as the time derivative of the spatial coordinate, which will coincide with the time derivative of the ALE map:
\begin{equation}
    w(x,t) = \pdv{x}{t} = \pdv{\alemap}{t} \ (x, t) = \pdv{\alemap}{t} \ (\alemap^{-1} (\aleX,t), t).
\end{equation}

\subsubsection*{Time-Derivative in the Reference Domain}
Since the equations are going to be solved in the physical domain, we need to adjust the time derivative in the physical domain 
$\left(\left.\frac{\partial u}{\partial t}\right|_{x}\right)$, 
so that it takes into account the movement of the mesh nodes.
By application of the chain rule we get
\begin{equation}
    \left.\frac{\partial u}{\partial t}\right|_{\aleX} = 
    \left.\frac{\partial u}{\partial t}\right|_{x} + w \pdv{u}{x},
\end{equation}
from where we get the necessary modification to be done to the strong form (\ref{eq:1d_fom_strong_pde}) of the PDE, 
\begin{equation}
    \left.\frac{\partial u}{\partial t}\right|_{\aleX} 
    + \left(\frac{\gamma+1}{2}\right) u \frac{\partial u}{\partial x} 
    - \left(a_0 + w\right)\frac{\partial u}{\partial x} 
    - \varepsilon \pdv[2]{u}{x} = 0.
\end{equation}
An additional convective term shows up, to take into account the movement of the nodes.
We will show in the results section how, if we remove this term, mass is no longer conserved.
\mytodo{Quote paper with FSI piston.}
\mytodo{Define ALE map and mesh velocity for the piston problem.}

% -----------------------------------------------------------------------------
\subsubsection{Nondimensional Equations}
Finally, as it is a good practice, 
we carry out a non-dimensionalization of the velocity and the spatial coordinate,
\begin{subequations}
    \begin{align}
        \tilde{x} = \frac{x}{L_0}, \qquad \tilde{u} = \frac{u}{a_0},
    \end{align}
\end{subequations}
with respect to the static speed of sound~$a_0$ and the piston's initial length~$L_0$.
This leads to the PDE and boundary condition
\begin{subequations}
    \begin{align}
        \pdv{\tilde{u}}{t} 
        + \left(\frac{\gamma+1}{2}\right)\left(\frac{a_0}{L_0}\right) \tilde{u} \pdv{\tilde{u}}{\tilde{x}}
        - \left(\frac{\varepsilon}{L_0^2}\right) \pdv[2]{\tilde{u}}{\tilde{x}}
        \nonumber
        \\[2mm]
        - \left(\frac{a_0}{L_0}\right)\left(1 + \frac{w}{a_0}\right)\pdv{\tilde{u}}{\tilde{x}}
         &= 0, 
        \\[2mm]
        \tilde{u}(L(t),t) = \frac{L'(t)}{a_0} &= b_L(t).
    \end{align}    
\end{subequations}
For the sake of clear notation, we condense the coefficients 
% which multiply the function and its derivatives,
\begin{subequations}
    \begin{align}
        b_0    &= \frac{a_0}{L_0} \left(\frac{\gamma+1}{2}\right), 
        \\[2mm]
        b_L(t) &= - \frac{\delta L_0 \omega}{a_0} \sin(\omega t),
        \\[2mm]
        c(x,t) &= \left(\frac{a_0}{L_0}\right)\left(1 - \frac{w}{a_0}\right),
    \end{align}
\end{subequations}
and drop the $\tilde{u}$ notation to prevent an overloaded notation.

The coefficient $\left(\frac{\delta L_0 \omega}{a_0}\right)$ multiplying the boundary condition
is the relation between the piston motion and the static speed of sound\footnote{a proxy for the average speed at which information propagates.}.
As we will see in the results section, this coefficient will be the one driving the response of the fluid to the motion of the piston.
The larger it is, the stronger the nonlinear response will become. 
We shall exploit this fact to build our reduced space, by sampling smartly through the parameter space.

With all of this done, we obtain a familiar PDE structure with diffusion, linear and non-linear convection terms:
\begin{equation}
    \left.\frac{\partial u}{\partial t}\right|_{\aleX} 
    + b_0 u \frac{\partial u}{\partial x} 
    - c(x,t)\frac{\partial u}{\partial x} 
    - \varepsilon  \pdv[2]{u}{x} = 0.
\end{equation}
Before moving on to the derivation of the weak formulation, we apply the scaling to the mass conservation and derived variables expressions.

\subsubsection{Mass Conservation}
If this rescaling is applied, Equation~(\ref{eq:mass_conservation_check}) for mass conservation needs to be updated, 
to take into account the fact that the velocity variable is now scaled with the speed of sound.
\begin{subequations}
    \begin{align}
        I(t) &= \int_{0}^{L(t)} \left(1 - \frac{\gamma-1}{2}u(x,t)\right)^{\frac{2}{\gamma-1}} dx, 
        \\
        MC(t) &= I'(t) - a_0 u(0,t) \left(1 - \frac{\gamma-1}{2}u(0,t)\right)^{\frac{2}{\gamma-1}}.
    \end{align}    
\end{subequations}

\subsubsection{Derived Variables}
The derived magnitudes of interest, Equations~(\ref{eq:channel_pressure_gradient}) and~(\ref{eq:outflow_at_the_boundary}), have to be rescaled too:
The outflow at the boundary $F_L(t)$ is given by
\begin{equation}
    F_L(t) = a_0 \rho_0 u(0,t) \rho(0,t).
\end{equation}

The pressure gradient across the channel is the difference between pressure at the piston and at the outflow boundary,
\begin{equation}
    \Delta_L p(t) = - \frac{p(0,t) - p(L(t),t)}{L(t)}. 
\end{equation}
We have defined it so that when the piston is moving forward, pushing the gas out of the chamber, the piston pressure gradient is defined positive.
% \subsubsection*{Geometric Conservation Laws}

\subsubsection{Weak Formulation}
Since we will be working with the Galerkin procedure to solve PDEs, 
we define the $L^2(\Omega)$ inner product to transform the strong formulation into a weak, variational one,
\begin{equation}
    %\inner{u}{v}_{(t, \mu)} = \int_{\Omega(t, \mu)} uv\ d\Omega.
    \inner{u}{v} = \int uv\ d\Omega.
\end{equation}
This inner product induces the so called \emph{eyeball} norm, since it checks if two functions look alike,
\begin{equation}
    % \norm{f-g}_{(t, \mu)}  = \sqrt{\int_{\Omega(t, \mu)} (f - g)^2\ d\Omega}.
    \norm{f-g}  = \sqrt{\int (f - g)^2\ d\Omega}.
\end{equation}
Eventually, we will also be interested in other norms, such as the $H^1(\Omega)$ norm, which captures the differences in the gradients too.
This is important when computing stress-related values from the solved field.

With this inner product, we project the residual of the strong formulation onto a given fuction $v \in V$,
where $V$ is a suitable Hilbert space,
\mytodo{Is this the right way to formulate a weak form?}
\begin{subequations}
    \begin{align}
        \label{eq:1d_fom_weak_formulation}
        \inner{\frac{\partial u}{\partial t}}{v} 
        \textcolor{NavyBlue}{+ \inner{b_0 u \grad u}{v}}
        \nonumber \\ 
        \textcolor{Purple}{- \inner{c \grad u}{v}} 
        \textcolor{OliveGreen}{+ \inner{\varepsilon \grad u}{\grad v}} 
        &= 0,  
        \\
        u(x, 0) &= 0, 
        \\
        u(L, t) &= b_L(t), 
        \\
        b_L(t) &= - \frac{\delta L_0 \omega}{a_0} \sin \omega t.
    \end{align}
\end{subequations}
% The outflow Neumann condition has been taken into account, but since it is homogeneously zero, it does not show up. 
The natural boundary condition, which for this problems translates into an homogeneous Neumann condition,
has already been taken into account in the integration by parts procedure of the diffusion term. 

\subsubsection{Dirichlet Lifting}
\label{sec:1d_fom_heat_equation_dirichlet_lifting}
For reasons that will become apparent later, it is preferable to work with a homogeneous problem in the Dirichlet boundary conditions.

To obtain so, we introduce a \emph{lifting} $g(x,t)$ of the Dirichlet boundary conditions.
We express the solution of our problem like the linear combination of the solution of the homogeneous problem and the lifting function:
\begin{equation}
    \label{eq:1d_fom_homogeneous_plus_lifting}
    u(x,t) = \hat{u}(x,t) + g(x,t).
\end{equation}
There are two conditions to be met by the lifting of the boundary conditions:
\begin{enumerate}
    \item To reach the prescribed values at the boundary nodes.
    \item To be sufficiently smooth within the domain.
\end{enumerate}
In a one-dimensional setting, the definition of a lifting function $g(x,t)$ is straigtforward, with a simple linear interpolation of the boundary values:
\begin{equation}
    \label{eq:1d_fom_dirichlet_lifting}
    g(x,t) = b_L(t) \left(\frac{x}{L}\right)
\end{equation}
% We verify that the function reaches the prescribed values at the boundaries,
% \begin{subequations}
%     \begin{align*}
%         g(0,t) &= b_L(t) \left(\frac{0}{L}\right) + b_0(t) \left(\frac{L}{L}\right) = b_0(t), \\
%         g(L,t) &= b_L(t) \left(\frac{L}{L}\right) + b_0(t) \left(\frac{L - L}{L}\right) = b_L(t).
%     \end{align*}
% \end{subequations}
% The smoothness condition is reached thanks to the linearity and smoothness of the interpolation functions.

In higher dimensional settings this procedure is valid too. 
However, due to the arbitrary shape the domain can take, the construction of the lifting function becomes more laborious.
We shall skip that for the moment, since we are dealing with a one-dimensional problem, where we can build analytically the extension of the boundary conditions. 

Introducing the lifting breakdown \eqref{eq:1d_fom_homogeneous_plus_lifting} into the weak formulation \eqref{eq:1d_fom_weak_formulation},
we find additional forcing terms and PDE terms, due to the cross-product between the function and its derivative, 
\begin{equation}
    u \pdv{u}{x} = \left(\hat{u} + g\right)\left(\pdv{\hat{u}}{x} + \pdv{g}{x}\right).
\end{equation}
Taking this into account, we find the following \textit{lifted} weak formulation,
\begin{subequations}
    % \label{eq:1d_fom_weak_formulation_homogeneous}
    \begin{align}
        \inner{\frac{\partial \hat{u}}{\partial t}}{v} \textcolor{Purple}{- \inner{c \grad \hat{u}}{v}} \textcolor{OliveGreen}{+ \inner{\varepsilon \grad \hat{u}}{\grad v}} \nonumber\\
        \textcolor{NavyBlue}{+ \inner{b_0 g \grad \hat{u}}{v}
        + \inner{b_0 \hat{u} \grad g}{v}} \nonumber\\
        \textcolor{NavyBlue}{+ \inner{b_0 \hat{u} \grad \hat{u}}{v}} \nonumber\\
        = 
        - \inner{\frac{\partial g}{\partial t}}{v} \textcolor{Purple}{+ \inner{c \grad g}{v}} \textcolor{OliveGreen}{- \inner{\varepsilon \grad g}{\grad v}} \nonumber\\ 
        \textcolor{NavyBlue}{- \inner{b_0 g \grad g}{v}},
    \end{align}
\end{subequations}
where we have reorganized the terms to show in order: linear dependency with the solution, terms due to cross-product effects, and finally the actual non-linearity.

Thanks to the lifting, we find the homogenization of the boundary conditions for all time $t$,
\begin{subequations}
    \begin{align}
        \hat{u}(L, t) = 0,
    \end{align}
\end{subequations}
and recall that the initial condition should be modified accordingly to the homogeneous problem definition, 
\begin{subequations}
    \begin{align}
    u(x,0) &= \hat{u}(x,0) + g(x,0), \\ 
    \hat{u}_0(x):= \hat{u}(x,0) &= u(x,0) - g(x,0). 
    \end{align}
\end{subequations}
At this point, we have defined the continuous problem for the one-dimensional piston in a moving domain.
\mytodo{Did I modify correctly my initial condition in the code?}

\subsection{Semi-Discrete Formulation: Time-Discretization}
The continuous solution changes in two dimensions: space and time.
Eventually, we need to discretize them both, but we can do so incrementally.

First, we discretize in time. 
To do so, we need to build an approximation for the time derivate, $\left(\frac{\partial \hat{u}}{\partial t}\right)$.
There is a complete body of knowledge devoted to this step of the numerical scheme.
\mytodo{Cite works on time discretization.} 

Briefly, there is a trade-off between computational effort and stability in terms of the resulting matrices.
For our needs, we opt for the maximization of stability, so we chose what is called an semi-implicit scheme, 
which is potentially stable, in exchange for a more dense linear system to be solved.

The procedure is the following: a polynomial interpolation of the function $u(x,t)$ is built, 
using the solution at previous timesteps, $\left\{u(x,t^{n}), u(x,t^{n-1}), \ldots\right\}$.
Then, the interpolant's derivative at time $t^{n+1}$ is used as an approximation of the actual derivative.
The accuracy of the scheme (and also its complexity) is determined by the number of previous timesteps used in the interpolation. 

In the coming sections, we define two implicit schemes of order one and two, BDF-1 (also known as Backwards Euler), and BDF-2, respectively.
Throughout our simulations we will use the BDF-2 scheme, but since it uses two points from the past, 
it cannot be used at the beginning of the simulation. 
That is where the BDF-1 comes into play, to obtain $u(x,t^1)$.
Additionally, studying the BDF-1 scheme is still useful, to understand the implications of time discretization. 
In terms of notation, from now on we use
\begin{equation}
    \hat{u}^n := \hat{u}(x,t^n)
\end{equation}
to define a function in space evaluated at time $t^n$.

\subsubsection{Semi-Implicit Scheme}
As a final note, we point out that according to each problem needs or complexities, sometimes a convergent scheme can be obtained even if certain equation terms are treated explicitly.
This trick can be used in problems such as the Navier-Stokes equations, where going fully implicit leads to a non-linear algebraic system. 
If the nonlinear convective term is treated semi-implicitly, one recovers a linear system, and yet reaches a satisfactory numerical solution.

Additionally, as we shall see further down in the reduced model, we will be able to assemble the reduced non-linear operator with the MDEIM procedure.
If for the evaluation of the nonlinear we used $u^{n+1}$, we would have a non-linear system which could be solved with a Newton-Rapshon iterative method.
Although this method is certainly the most robust, it is not efficient. 

Thus, we opt for the approximation of its effect by considering the \textit{convection velocity} at time $t$, instead of time $t^{n+1}$,
\begin{equation}
    \inner{b_0 \hat{u} \grad \hat{u}}{v}^{n+1} \simeq \inner{b_0 \hat{u}^{n} \grad \hat{u}^{n+1}}{v}.
\end{equation}
With this strategy we obtain a semi-implicit scheme, which can be treated as a linear system.

\subsubsection{BDF-1}
\mytodo{Update BDF-1 proof, it is done with the heat equation. Should I move this to an appendix? Should I simplify this to be equation-independent?}
To derive the BDF-1 scheme, we start by evaluating our weak formulation at time $t^{n+1}$, where the solution is unknown,
\begin{equation}
    \begin{split}
        \inner{\frac{\partial \hat{u}}{\partial t}}{v}^{n+1} 
        - \inner{w \grad \hat{u}}{v}^{n+1} 
        + \inner{\varepsilon \grad \hat{u}}{\grad v}^{n+1} =  \\
        + \inner{f}{v}^{n+1} + \inner{f_{g,1}}{v}^{n+1} + \inner{f_{g,2}}{\grad v}^{n+1}.
    \end{split}
\end{equation}
Then, we construct our interpolation polynomial using only one previous timestep from the past, 
\begin{align}
    \hat{u}(x,t) &\simeq I_1(t) := 
    \hat{u}^{n+1}\left( \frac{t-t^{n}}{\dt}  \right)
    + \hat{u}^{n}\left(\frac{t^{n+1}-t}{\dt}\right), \\
    \left.\frac{\partial\hat{u}}{\partial t}\right|^{n+1} &\simeq \frac{d I_1(t)}{dt} = \frac{\hat{u}^{n+1} - \hat{u}^{n}}{\dt}
\end{align}
With this approximation of the time derivative, we get the following semi-discrete weak formulation, 
% \begin{subequations}
%     \begin{align}
%         \label{eq:1d_fom_weak_formulation_semidiscrete_time}
%         \inner{\hat{u}^{n+1}}{v} 
%         - \dt \inner{w \grad \hat{u}^{n+1}}{v} 
%         + \dt \inner{\varepsilon \grad \hat{u}^{n+1}}{\grad v} = \nonumber \\ 
%         \inner{\hat{u}^{n}}{v} 
%         + \dt \inner{f^{n+1}}{v} \nonumber \\
%         + \dt \inner{f_{g,1}^{n+1}}{v} + \dt \inner{f_{g,2}^{n+1}}{\grad v}, \\
%         \hat{u}^{n+1}(0) = 0, \\
%         \hat{u}^{n+1}(L\left(t^{n+1}\right)) = 0, \\
%         \hat{u}^{0}(x) = \hat{u}_0(x).
%     \end{align}
% \end{subequations}
\begin{subequations}
    \label{eq:1d_fom_weak_formulation_homogeneous}
    \begin{align}
        \inner{\hat{u}^{n+1}}{v} 
        - \dt \inner{c \grad \hat{u}^{n+1}}{v} 
        + \dt \inner{\varepsilon \grad \hat{u}^{n+1}}{\grad v} \nonumber\\
        + \dt\left(\inner{b_0 \hat{u}^{n+1} \grad g^{n+1}}{v}\right. 
        + \left.\inner{b_0 g^{n+1} \grad \hat{u}^{n+1}}{v}\right) \nonumber\\
        + \inner{b_0 \hat{u} \grad \hat{u}}{v}^{n+1} \nonumber\\
        = 
        \inner{\hat{u}^{n}}{v} 
        - \dt \inner{\frac{\partial g^{n+1}}{\partial t}}{v} \nonumber \\
        - \dt \inner{b_0 g^{n+1} \grad g^{n+1}}{v} 
        - \dt \inner{\varepsilon \grad g^{n+1}}{\grad v} \nonumber\\ 
        + \dt \inner{c^{n+1} \grad g^{n+1}}{v}
    \end{align}
\end{subequations}

\mytodo{Include BDF-2 discrete scheme.}

Note that the problem is still continuous in space, but no longer in time.

% \subsubsection{BDF-2}
% To derive the BDF-2 scheme, again, we start by evaluating our weak formulation at time $t^{n+1}$, where the solution is unknown,
% \begin{equation}
%     \begin{split}
%         \inner{\frac{\partial \hat{u}}{\partial t}}{v}^{n+1} + \inner{\varepsilon(x) \grad \hat{u}}{\grad v}^{n+1} = \inner{f}{v}^{n+1} \\
%         + \inner{f_{g,1}}{v}^{n+1} + \inner{f_{g,2}}{\grad v}^{n+1}.
%     \end{split}
% \end{equation}
% Then, we construct our interpolation polynomial using two previous timestamps from the past, 
% \begin{align}
%     \hat{u}(x,t) &\simeq I_2(t) := 
%       \varepsilon_1 \hat{u}^{n+1} 
%     + \varepsilon_2 \hat{u}^{n}
%     + \varepsilon_3 \hat{u}^{n-1}, \\
%     \left.\frac{\partial\hat{u}}{\partial t}\right|^{n+1} &\simeq \frac{d I_2(t)}{dt} = 
%     \frac{\frac{3}{4}\hat{u}^{n+1} - \frac{1}{2}\hat{u}^{n} + \frac{1}{4}\hat{u}^{n-1}}{\dt}
% \end{align}
% With this approximation of the time derivative, we get the following semi-discrete weak formulation, 
% \begin{subequations}
%     \begin{align}
%         \label{eq:1d_fom_weak_formulation_semidiscrete_time_bdf2}
%         \frac{3}{4}\inner{\hat{u}^{n+1}}{v} + \dt \inner{\varepsilon(x) \grad \hat{u}^{n+1}}{\grad v} &= \\
%           \frac{1}{2}\inner{\hat{u}^{n}}{v}
%         - \frac{1}{4}\inner{\hat{u}^{n-1}}{v} \nonumber \\
%         + \dt \inner{f^{n+1}}{v} \nonumber \\
%         + \dt \inner{f_{g,1}^{n+1}}{v} + \dt \inner{f_{g,2}^{n+1}}{\grad v}, \nonumber \\
%         \hat{u}^{n+1}(0) &= 0, \\
%         \hat{u}^{n+1}(L\left(t^{n+1}\right)) &= 0, \\
%         \hat{u}^{0}(x) &= \hat{u}_0(x).
%     \end{align}
% \end{subequations}
% Note how the only changes between both discretizations are the forcing terms due to the previous solutions and the leading coefficient multipliying the inner product of the desired unknown solution $\hat{u}^{n+1}$. 

\subsection{Discrete Problem: Space and Time Discretization}
To complete our discretization, we define a finite functional space $V_h \subset V$, where we can represent the solution as the linear combination of a set of Finite Elements (FE) basis functions $\varphi_i(x)$ with local support,
\begin{align}
    \hat{u}^{n}(x) \simeq \hat{u}_h^n(x) &= \sum_j^{N_h} \hat{u}_{h_j}^n \varphi_j(x), \\
    \vb{\hat{u}}_h^{n} &= \left[\hat{u}_{h_j}^n\right].
\end{align}
In this discrete setting, we define the FE vector $\vb{\hat{u}}_h^{n}$ to be the collection of coefficients~$[\hat{u}_{h_j}^n]$ which multiply the basis functions.
In the FE context, these coincide with the values of the function at each node.

Applying the Galerkin principle to solve PDEs, we enforce the orthogonality of the residual to the functional space $V_h$. 
%This is equivalent to enforcing the orthogonality to each basis function of the space, 
%so we get an algebraic system with the same number of unknowns as equations.
Because the domain changes with time, 
% even if we are solving a linear system, 
both the matrices and the vectors change for each timestep,
\begin{subequations}
    \label{eq:1d_fom_linear_system_operators}
    \begin{align}
        \left[\vb{M}_{h}^{n+1}\right]_{ij}           &= m_{\text{BDF}} \inner{\varphi_j}{\varphi_i}, 
        \\
        \left[\vb{A}_{h}^{n+1}\right]_{ij}           &= \inner{\varepsilon \grad \varphi_j}{\grad \varphi_i}, 
        \\
        \left[\vb{C}_{h}^{n+1}\right]_{ij}           &= -\inner{c \grad \varphi_j}{\varphi_i}, 
        \\
        \left[\vb{N}_{h}^{n+1}\right]_{ij}           &= b_0 \inner{u^{n} \grad \varphi_j}{\varphi_i}, 
        \\
        \left[\vb{\hat{N}}_{h}^{n+1}\right]_{ij}     &= b_0 \left(\inner{g \grad \varphi_j}{\varphi_i} + \inner{\varphi_j \grad g}{\varphi_i}\right), 
        \\
        \left[\vb{F}_{{g,h}}^{n+1}\right]_{i}        &= -\inner{\pdv{g}{t} + b_0 g \grad g - c \grad g}{\varphi_i} \nonumber
        \\
                                                     &  -\inner{\varepsilon \grad g}{\grad \varphi_i},
        \label{eq:1d_fom_linear_system_operators_fg}
        \\ 
        \left[\vb{F}_{\vb{\hat{u}}_h}^{n}\right]_{i} &= 
            \begin{cases}
                \inner{\hat{u}_h^{n}}{\varphi_i}, &\text{BDF-1},
                \\
                \frac{1}{2}\inner{\hat{u}_h^{n}}{\varphi_i}
                - \frac{3}{4}\inner{\hat{u}_h^{n-1}}{\varphi_i}, &\text{BDF-2}.
            \end{cases}
    \end{align}
\end{subequations}
This leads to the following algebraic system:
\begin{subequations}
    \begin{align}
        \label{eq:1d_fom_weak_formulation_discrete}
        m_{\text{BDF}} \vb{M}^{n+1}_h \vb{\hat{u}}_{h}^{n+1} 
        + \dt \vb{C}^{n+1}_h \vb{\hat{u}}_{h}^{n+1} 
        + \dt \vb{A}^{n+1}_h \vb{\hat{u}}_{h}^{n+1} 
        \nonumber 
        \\ 
        + \dt \vb{\hat{N}}^{n+1}_h \vb{\hat{u}}_{h}^{n+1} 
        + \dt \left[\vb{N}^{n+1}_h\left(\vb{\hat{u}}_{h}^{n}\right)\right] \vb{\hat{u}}_{h}^{n+1} 
        \nonumber
        \\ 
        = \vb{F}_{\vb{\hat{u}}_h}^{n}
        + \dt \vb{F}_{g,h}^{n+1}, \\
        \vb{\hat{u}}_h^{0} = \vb{\hat{u}}_{h,0}.
    \end{align}
\end{subequations}
The spatial boundary conditions are encoded within the matrices and the vectors. 
The initial condition is obtain via interpolation or projection.
The assembly of the forcing terms given in Equation \eqref{eq:1d_fom_linear_system_operators_fg}
are to be done with the FE vector representations of the functional expressions of each of the terms $f$ and $f_g$, to be obtained by projection or interpolation.

The parameter $m_{\text{BDF}}$ determines the time integration scheme used,
\begin{equation}
    m_{\text{BDF}} = 
    \begin{cases}
    1, &\text{BDF-1},
    \\    
    \frac{3}{4}, &\text{BDF-2}.
    \end{cases}
\end{equation}
Regarding the forcing due to previous timesteps, $\left(\vb{F}_{\vb{\hat{u}}_h}^{n}\right)$, 
although for the FOM model we could compute the inner products at each timestep, 
for the Reduced Order Model we will exploit an algebraic expression of these expressions.
It can be expressed as the product between the mass matrix and the FE representation of the previous solution(s), 
\begin{equation}
    \label{eq:1d_fom_forcing_term_time_mass_matrix}
    \vb{F}_{\vb{\hat{u}}_h}^{n}= 
    \begin{cases}
        \vb{M}_{h}^{n+1} \vb{\hat{u}}_{h}^{n},                &\text{BDF-1},
        \\
        \frac{1}{2}\vb{M}_{h}^{n+1} \vb{\hat{u}}_{h}^{n}
        - \frac{3}{4}\vb{M}_{h}^{n+1} \vb{\hat{u}}_{h}^{n-1}, &\text{BDF-2}.
    \end{cases}
\end{equation}
We point out how the timestep $\dt$ has been intentionally left out of the discrete operators definition.
There are two reasons to back this decision:
\begin{enumerate}
    \item Conceptually, each discrete operator encodes a spatial model, in terms of a differential operator or the presence of a forcing term.
    The timestep $\dt$ shows up because we first discretized the continuous problem in time. 
    Had we gone the other way around, discretizing the problem in space in the first place, we would have found a system of ODEs with the previously defined spatial operators. 
    \item When we leverage the system approximation reduction technique, we will not want to have there the presence of the timestep.
    The reduced model could use a different timestep, or the snapshots for different $\mu$ values could be collected for different timestep values.
\end{enumerate}
% The computation of the discrete initial condition $\vb{\hat{u}}_{h,0}$ is addressed in Section~\ref{sec:1d_fom_projection_interpolation}.

If we collect terms and factor out the unknowns, we get a compact linear system to be solved at each timestep to advance the solution,
\begin{subequations}
    \begin{align}
        \label{eq:1d_fom_linear_system_timestep}
        \vb{K}_h^{n+1} \vb{\hat{u}}_h^{n+1} &= \vb{b}_h^{n+1}, 
        \\
        \vb{\hat{u}}_h^{0} &= \vb{\hat{u}}_{h,0},
        \\
        \vb{K}_h^{n+1} &= \Ah{M} + \dt \left[\Ah{A} + \Ah{B} \right. 
        \nonumber 
        \\
                        &\left. + \Ah{\hat{N}} + \Ah{N}\left(\vb{\hat{u}}_{h}^{n}\right)\right],
        \\
        \vb{b}_h^{n+1} &= \vb{F}_{\vb{\hat{u}}_h}^{n} + \dt \vb{F}_{g,h}^{n+1}.
    \end{align}
\end{subequations}

% \subsubsection{FE Representation of the Initial Condition}
% \label{sec:1d_fom_projection_interpolation}
% The initial condition in time $\hat{u}_0(x)$ needs to be expressed in terms of the basis functions $\varphi_i(x)$ to obtain the FE vector~$\vb{\hat{u}}_{h,0}$.
% There are two ways to compute the values of this FE vector:
% \begin{itemize}
%     \item Interpolation.
%     \item Projection.
% \end{itemize}

% \subsubsection*{Interpolation}
% Since we are dealing with a nodal basis\footnote{We recall a nodal basis is one where each node of the mesh has a basis function assigned, 
% such that the value of the coefficient which multiplies each basis function corresponds to the value of the function at that node.}, 
% the interpolation procedure simply assigns to each vector entry the value of the function at the node corresponding to that entry.

% In a sense, it is what we have done with the lifting of the boundary conditions in Section~\ref{sec:1d_fom_heat_equation_dirichlet_lifting}.
% Since we are dealing with a nodal basis, we can express the lifting FE vector~$\vb{g}_h$ as a linear combination of the FE basis functions,
% \begin{equation}
%     g(x,t) \simeq g_h(x,t) = \sum_i^{N_h} g_{h_i}(t) \varphi_i(x),
% \end{equation}
% where each coefficient $\vb{g}_h = [g_{h_i}]$ is the value taken by the analytical interpolation of the boundary conditions given in Equation~\eqref{eq:1d_fom_dirichlet_lifting}. 

% \subsubsection*{Projection}
% Instead, the projection procedure aims at solving the approximation problem
% \begin{equation}
%     \hat{u}^{0}_h(x) = \sum_j^{N_h} \hat{u}_{h_i}^0 \varphi_j(x) = \hat{u}_0(x)
% \end{equation}
% with a variational approach within $V_h$. 
% That is, the approximation error is set to be orthogonal to the function space $V_h$, yielding the weak formulation
% \begin{equation}
%     \inner{\hat{u}^{0}_h(x)}{v} = \inner{\hat{u}_0(x)}{v}, \, \forall \, v \in V_h,
% \end{equation}
% which leads to a linear system of equations if we enforce the orthogonality to each basis function,
% \begin{subequations}
%     \begin{align}
%         \label{eq:projection_equation}
%         \vb{M}_{h} \vb{\hat{u}}_{h,0} &= \vb{p}_h, \\
%         \left[\vb{M}_{h}\right]_{ij} &= \inner{\varphi_j}{\varphi_i}, \\
%         \left[\vb{p}_{h}\right]_{i} &= \inner{\hat{u}_0(x)}{\varphi_i}.
%     \end{align}
% \end{subequations}

% In practice, the projection method is to be used, because even within the FE context we might not have a nodal basis (this would be the case if exotic function spaces are used).

% However, when the Method of Manufactured Solutions is being used to certify the code implementation, only the interpolation procedure will allow us to reach machine accuracy when computing the error.

\subsubsection{Wind-Up for the Full Order Problem}
With all of the above, the generic Finite Element Method for the one-dimensional piston problem is defined.

The problem has been explained within three levels of abstraction: continuous with strong and weak formulations, semi-discrete in time and fully discretized in time and space.
An semi-implicit time discretization scheme is used, to obtain a sufficiently stable algebraic system.
\mytodo{How does the semi-discretization play with the stability of the system?} 

A lifting of the Dirichlet boundary conditions has been introduced, which lead to the appearence of an additional forcing term and the homogenization of the boundary conditions.
Hence, we will focus in the solution of an homogeneous boundary value problem.
This fact will prove useful when we get into the implementation details of the reduction procedure.

\subsection{Deformation of the Space-Time Domain}
TBD.

Comment on how in the $(x,t)$ domain we actually have a deforming domain. 

This could be solved with space-time FE, but it involves complex tensor products. 

Explain how what we have done is a good approximation of the latter.

\subsection{Generalized Transformation}
If the domain is made fixed, 
how does the equation change?
What would be the expression of the algebraic system?

\end{document}