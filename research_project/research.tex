%% Lab Report for EEET2493_labreport_template.tex
%% V1.0
%% 2019/01/16
%% This is the template for a Lab report following an IEEE paper. Modified by Francisco Tovar after Michael Sheel original document.


%% This is a skeleton file demonstrating the use of IEEEtran.cls
%% (requires IEEEtran.cls version 1.8b or later) with an IEEE
%% journal paper.
%%
%% Support sites:
%% http://www.michaelshell.org/tex/ieeetran/
%% http://www.ctan.org/pkg/ieeetran
%% and
%% http://www.ieee.org/

%%*************************************************************************
%% Legal Notice:
%% This code is offered as-is without any warranty either expressed or
%% implied; without even the implied warranty of MERCHANTABILITY or
%% FITNESS FOR A PARTICULAR PURPOSE! 
%% User assumes all risk.
%% In no event shall the IEEE or any contributor to this code be liable for
%% any damages or losses, including, but not limited to, incidental,
%% consequential, or any other damages, resulting from the use or misuse
%% of any information contained here.
%%
%% All comments are the opinions of their respective authors and are not
%% necessarily endorsed by the IEEE.
%%
%% This work is distributed under the LaTeX Project Public License (LPPL)
%% ( http://www.latex-project.org/ ) version 1.3, and may be freely used,
%% distributed and modified. A copy of the LPPL, version 1.3, is included
%% in the base LaTeX documentation of all distributions of LaTeX released
%% 2003/12/01 or later.
%% Retain all contribution notices and credits.
%% ** Modified files should be clearly indicated as such, including  **
%% ** renaming them and changing author support contact information. **
%%*************************************************************************

% \documentclass[a4paper, technote, compsoc]{IEEEtran}
\documentclass[../main.tex]{subfiles}

% \usepackage[T1]{fontenc}    % use 8-bit T1 fonts
% \usepackage[utf8]{inputenc} % allow utf-8 input
% \usepackage{amsmath}
% \usepackage{amssymb}
% \usepackage{booktabs}
% \usepackage{float}  % used to fix location of images i.e.\begin{figure}[H]
% \usepackage{graphicx}  %needed to include png, eps figures
% \usepackage{mwe}
% \usepackage{subfiles}
% \usepackage{url} % correct bad hyphenation here
% \usepackage{xcolor}

% \usepackage[backend=biber,style=ieee]{biblatex} 
% \bibliography{biblio.bib} %your file created using JabRef

\begin{document}

% % paper title
% \title{Reduced Order Models \\ With Moving Domains \\ \normalsize{Research Project}}

% % author names 
% \author{Enrique Millán Valbuena \\ \normalsize{463 426 8}}% <-this % stops a space
        
% % The report headers
% \markboth{M. Sc. Aerospace Engineering, TU Delft}%do not delete next lines
% {Shell \MakeLowercase{\textit{et al.}}: Bare Demo of IEEEtran.cls for IEEE Journals}

% % make the title area
% \maketitle

% % As a general rule, do not put math, special symbols or citations
% % in the abstract or keywords.
% \begin{abstract}
%    The research objective is to build a Reduced Order Model (ROM) for a two-dimensional parametrized unsteady PDE with a moving boundary:
%    \begin{itemize}
%       \item Heat diffusion problem.
%       \item (Bonus) Graetz convection-diffusion problem.
%    \end{itemize}
%    Both the main body of the PDE and the geometrical definition of the moving boundary will be parametrized.

%    A concise description of the reducing procedure is provided, together with a priori convergence rates for basis size estimation and a posteriori error estimators to certify the use of the Reduced Order Model.
%    Numerical examples to showcase computational costs and implementation details are designed, implemented and validated with the Manufactured Solutions Method. 
% \end{abstract}

% \begin{IEEEkeywords}
% Reduced Order Model, Moving Domain, FEM, DEIM, POD, Galerkin-Projection
% \end{IEEEkeywords}

\section{Introduction}
The MSc Thesis focuses in the fast solution of parametrized linear parabolic PDEs with moving boundaries.
For the sake of clarity, we shall now state the problem without too much mathematical formalism, which we shall keep for later.

Parametrized PDEs can be numerically solved with the Finite Element Method (FEM), which lead to an algebraic system of equations whose solution can be computationally expensive to obtain, specially for complex geometries or detailed models.
When this is the case, many-query procedures and access to field values or calculated outputs for different parameter values $\mu$ can become cumbersome, or even infeasible due to computational costs, both in time and memory.
One often refers to this FEM model as the \textit{Full Order Model}~(FOM),
\begin{equation*}
   \frac{du_h}{dt} + A_h\left(t;\mu\right) u_h = f_h\left(t;\mu\right).
\end{equation*}

To circumvent these issues, one can build a \textit{Reduced Order Model}~(ROM), whose solution is fast in time and light in storage.
This ROM is based in ad-hoc empirical basis functions, whose support typically spans the whole domain. 

\subsection{Brief Outline Of The Reduction Process}
The construction of the ROM has mainly two phases:
\begin{itemize}
   \item Offline phase: construction of the ROM.
   \item Online phase: usage of the ROM.
\end{itemize}

During the offline phase, the costly algebraic problem is solved for a subset of the parameter space.
Snapshots of the matrices, vectors and solutions are stored and processed via algebraic reduction algorithms, in order to obtain a reduced basis for each of them.

An example of reduction algorithms would be the Discrete Empirical Interpolation Method (DEIM) and its matrix version (M-DEIM).
These reduction algorithms need to rely on a procedure to construct the basis, in our case we will use the Proper Orthogonal Decomposition (POD). 

The offline phase scales with the dimension of the Full Order Model, $N_h$, which is governed by the number of nodes in the mesh and the polynomial degree for the FEM basis.
Once the offline phase is over, a basis for each operator of the algebraic problem has been produced, of representative size~$N \ll N_h$\footnote{
The reduction of each operator might have required a different number of basis elements, but they should be all of the same order of magnitude $N$ or smaller for the reduction process to be a success.}.
Additionally, from the solution snapshots a projection matrix is built
\begin{equation*}
   \mathbb{V} \in \mathbb{R}^{N_h \,\text{x}\, N}.
\end{equation*} 
This matrix will be used to obtain once and for all our ingredients to assemble the ROM.

These bases are used during the online phase, where for new parameters, different from those used in the offline phase, a small algebraic system is built and solved,
\begin{equation*}
   \frac{du_N}{dt} + A_N\left(t;\mu\right) u_N = f_N\left(t;\mu\right).
\end{equation*}
At this stage, it is paramount that the assembly of the operators\footnote{
   We shall abuse notation and refer to the operators for both the matrices and the functionals, unless explicit distinction is required.
} 
is independent of the original problem size $N_h$.
When this is the case, we state to have an \textit{
   ideal offline-online decoupling}.

An essential ingredient to achieve such decoupling is what we call an \textit{
   affine decomposition} 
of the problem operators.
Simply put, it is to say that we can achieve linear separation in the parameter and algebraic domains, 
\begin{equation*}
   A_h\left(t;\mu\right) = \sum_{q=1}^{Q} \Theta_{q}(t;\mu) A_{h,q},
\end{equation*}
where~$A_{h,q}$ are constant matrices and~$\Theta_{q}(t;\mu)$ are scalar values. 

The easiest example one could come up with of an affine decomposition is the one present in the heat diffusion problem with two different but constant diffusion parameters $k_q$ across the domain.
Then, the affine decomposition would look like
\begin{equation*}
   A_h\left(t;k_1, k_2\right) = k_{1} A_{h,1} + k_{2} A_{h,2},
\end{equation*}
where each matrix $A_{h,q}$ would represent the diffusion operator with support over the subdomain associated with each parameter. 
For this simple example, the affine decomposition is present naturally within the PDE structure, but this will not always be the case, specially when non-linearities are present. 

However, nowadays it is absolutely possible to obtain an automatic ad-hoc affine decomposition for any operator thanks to grounded algorithms and procedures, like DEIM and MDEIM.
This key fact will allows us to achieve a perfect split between the offline and the online phase, as it will allow us to assemble our ROM operators without having to assemble at any point the complete FOM operator. 

Finally, once the reduced problem has been solved, the solution can be projected back to the original physical mesh,
\begin{equation*}
   u_h = \mathbb{V} u_N.
\end{equation*}

Following this path, access to field values or calculated outputs can be obtained lightly, provided the overall procedure is \textit{certified}: 
to prove in the online phase that the solution is sufficiently close to what would have been if the actual FOM had been assembled and solved.
More on the certification process will be explained later.

% Ideally, 
%% Research objectives
% Useful: benefit of your research to the problem. 
% Realistic: contribute to the solution of the problem.
% Feasible: time scheduled and capabilities and resources.
% Clear: be precise in the contribution to the problem.
% Informative: rough idea of knowledge generated towards a solution.

% The research objective is (a) by (b).
% (a) The contribution of the research project to the solution of the problem.
% (b) description of the way the contribution will be provided. 

\begin{figure}[h]
   \includegraphics[width=\columnwidth]{research_project/figures/index.png}
   \caption{Conceptual location of the MSc. Thesis.}
\end{figure}

\subfile{research_project/details.tex}

\subsection{A Note On The Word \textit{Linear}}
A non-linearity is essentially a characteristic that prevents a linear separation scheme.

Despite the apparent linear character in algebra of the target equation, it is actually non-linear.
The introduction of the time variable~$t$~in the shape of the domain changes the operators at each time step during the integration procedure. 
Additionally, since the domain geometry will depend on some parameter values too, one cannot explicitly write the affine decomposition of the operators in a closed form.  

Another type of non-linearity would be a term whose value was a non-linear function of the field $u$ variable. 
These kind of non-linearities are common in mathematical modelling, but shall keep out of the scope of this work to keep a narrow focus. 
The non-linearity of the basic operators will give us enough work already.

\section{Objectives and Details}
Summing up, the research objective is to build a Reduced Order Model (ROM) for a two-dimensional parametrized unsteady PDE with a moving boundary:
\begin{itemize}
   \item Heat diffusion problem.
   \item (Bonus) Graetz convection-diffusion problem.
\end{itemize}
Both the main body of the PDE and the geometrical definition of the moving boundary will be parametrized.

We aim at providing a concise description of the reducing procedure, 
a posteriori error estimators to certify their use and 
a numerical example to showcase computational costs and implementation details.

Because the ROM can approach arbitrarily close the FOM, one would like to have a rough idea of how many reduced basis will be required to attain such accuracy. 
To that end, bounds for convergence rates of the reduction procedures must be obtained, and a posteriori error estimates of the reduction model to certify the solution stayed within the desired bracket. 

The numerical implementation will be validated with the manufactured solutions method.

\subsection{Scope}
The project is mainly practice-oriented, in that we shall design, build and evaluate the procedure to construct the ROM.

Nevertheless, we intend to have theoretical content too, as convergence rates and a posteriori error estimates require the deployment of theoretical development and theorems\footnote{
   Which rely on the usual stability factors and arguments present in the Finite Element body.} 
to find sharp bounds.

\subsection{Implementation Of The Moving Domain}
To capture the effects of a moving domain without unnecessarily complicating the implementation, we shall assume a separable deformation across the mesh, 
\begin{equation*}
   \vec{d}(t;\mu) = g(t) \vec{d}_s(\mu),
\end{equation*}
where the spatial deformation $\vec{d}_s(\mu)$ will be obtained from the solution of a simple Laplace problem over an initial undeformed domain $\Omega_0$,
\begin{align*}
   -\Delta \vec{d}_s &= \vec{0}, \quad \vec{x} \in \Omega_0, \\
   \vec{d}_s &= \vec{h}(\vec{x}, \mu) \in \partial\Omega_0.
\end{align*}
In this way, we obtain a smooth displacement across the domain, due to the harmonic properties of the Laplacian operator. 

For industrial applications, the elasticity equation ought to be solved instead, but this would only add complications to our problem which are out of scope for the reduction process we are concerned with.

This problem can be reduced too at the beginning of the offline phase, so that quicker displacements are obtained during the solution of the main PDE. 
The reduction technique for this problem is simpler, as it does not depend on time thanks to the separation assumption. 


%By introducing the construction of a posteriori error estimators, one can certify the results of the reduced basis for unseen in the training phase parameters.

%% Theory-oriented
% Theory development
% Theory testing

%% Practice-oriented: intervention cycle
% Problem Analysis
% Diagnosis
% Design
% Change
% Evaluation


%% Research Strategy
% Suitability
% Duration
% Feasibility

\section{Conclusions}
\label{sec:conclusions}
The conclusion of this Thesis will establish the reduction procedure for parametrized unsteady linear PDEs with a moving domain.

Thanks to the a posteriori error estimates, the procedure will be certified.
That is, the error between the FOM and the ROM will be proved to be under a given threshold. 

Thanks to the convergence rates, an estimate of the required bases for the overall procedure will be obtained a priori.
This helps with the scaling of the problem, which is an important attribute if the method was to be deployed for industrial purposes. 

The heat diffusion problem was used as a numerical example to showcase the orders of magnitude and the implementation details. 

(Bonus) Additionally, the Graetz convection-diffusion problem was solved too, to show the complications the presence of convective terms bring in to the reduction. 
\printbibliography

% http://danceswithcode.net/engineeringnotes/rotations_in_3d/rotations_in_3d_part1.html

\end{document}


