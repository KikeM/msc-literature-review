% 1.  Justifies the need for hyperreduction, and need for its application for moving domains.
% Reduced order models express the solution of a PDE as 
% the linear combination of so called RB solution modes.
% These modes have been identified with procedures which make them appropriate to capture
% the relevant dynamics encoded in the PDE (and its boundary conditions).

The algebraic system associated to RB solution modes is 
obtained by projection of the departing discretized operators,
assembled with conventional discretization techniques 
(finite elements, finite volumes, etc.).
If the RB solution basis size is small and sufficient to represent the solution,
the resulting matrices are often smaller by several orders of magnitude.

In the context of a fixed mesh, 
the assembly and projection steps take place once and for all.
On the contrary, if the mesh moves in time,
these steps need to take place for every time step.
Hence, despite the size reduction of the linear system,
there is still a considerable overhead during its assembly.
To overcome it, the (M)DEIM system approximation technique is introduced.
% The (M)DEIM technique defines an ad-hoc affine decomposition for each algebraic operator.
To use (M)DEIM, the operator is expressed as a linear combination of operator modes (collateral basis);
whose coefficients are obtaiend from an empirical interpolation method applied over
the evaluation of a restricted set of operator entries.
Since an ad-hoc basis is identified for the solution and each of the algebraic operators,
the combination of these two techniques defines an hyper reduced order model (HROM).

% 2 . Summarises the method you introduced and its results (something like the second paragraph of your report's conclusions). 
The methodology is purely algebraic, 
with a non-intrusive property that should allow its implementation on existing solvers.
We have obtained RB solution and operator bases;
to approximate the operators projection unto the reduced space 
without explicitly assembling and projecting the original FOM operator.
The resulting HROM bases can be truncated
to resolve for a determined error threshold with respect to the conventional model.
Building the collateral basis increases the cost of the offline stage,
but it allows a perfect offline-online split:
no FOM operators are used during the online run. 
For the operators, the offline stage is carried out separately from the FOM simulation,
so that a wider parameter range can be spanned. 

% We present a reduced order model (ROM) for a one-dimensional nonlinear gas dynamics problem:
% the isentropic piston.
% The main body of the PDE, 
% the geometrical definition of the mesh nodes, 
% and the boundary conditions are parametrized.
% The full order model is obtained with a Galerkin finite element discretization,
% under the Arbitrary Lagrangian Eulerian formulation (ALE).
% To stabilize the system, an artificial viscosity term is included.
% The nonlinear convective term is linearized with a second-order extrapolation.
% The reduced basis to express the solution is obtained with the POD technique.
% To overcome the explicit use of the Jacobian transformation, 
% typical in the context of moving meshes,
% a system approximation technique is introduced.
% The (Matrix) Discrete Empirical Interpolation Method, (M)DEIM, allows us
% to work with a weak form defined in the physical domain 
% (and hence the physical weak formulation)
% whilst maintaining an efficient assembly of the algebraic operators, 
% despite their change with every time step.
% Two alternative methods are presented to collect and compress the snapshots 
% for the linearized solution-dependent convective term.
% Each method leads to a different offline stage.
% All in all, our approach is purely algebraic
% and the reduced model makes no use of full order structures, 
% thus achieving a perfect \textit{offline-online} split.
% A concise description of the reduction procedure is provided.
% The reduced model is certified with a posteriori error estimations obtained via model truncation.