%% Lab Report for EEET2493_labreport_template.tex
%% V1.0
%% 2019/01/16
%% This is the template for a Lab report following an IEEE paper. Modified by Francisco Tovar after Michael Sheel original document.


%% This is a skeleton file demonstrating the use of IEEEtran.cls
%% (requires IEEEtran.cls version 1.8b or later) with an IEEE
%% journal paper.
%%
%% Support sites:
%% http://www.michaelshell.org/tex/ieeetran/
%% http://www.ctan.org/pkg/ieeetran
%% and
%% http://www.ieee.org/

%%*************************************************************************
%% Legal Notice:
%% This code is offered as-is without any warranty either expressed or
%% implied; without even the implied warranty of MERCHANTABILITY or
%% FITNESS FOR A PARTICULAR PURPOSE! 
%% User assumes all risk.
%% In no event shall the IEEE or any contributor to this code be liable for
%% any damages or losses, including, but not limited to, incidental,
%% consequential, or any other damages, resulting from the use or misuse
%% of any information contained here.
%%
%% All comments are the opinions of their respective authors and are not
%% necessarily endorsed by the IEEE.
%%
%% This work is distributed under the LaTeX Project Public License (LPPL)
%% ( http://www.latex-project.org/ ) version 1.3, and may be freely used,
%% distributed and modified. A copy of the LPPL, version 1.3, is included
%% in the base LaTeX documentation of all distributions of LaTeX released
%% 2003/12/01 or later.
%% Retain all contribution notices and credits.
%% ** Modified files should be clearly indicated as such, including  **
%% ** renaming them and changing author support contact information. **
%%*************************************************************************

% \documentclass[a4paper, technote, compsoc]{IEEEtran}
\documentclass[../main.tex]{subfiles}

% \usepackage[T1]{fontenc}    % use 8-bit T1 fonts
% \usepackage[utf8]{inputenc} % allow utf-8 input
% \usepackage{amsmath}
% \usepackage{amssymb}
% \usepackage{booktabs}
% \usepackage{float}  % used to fix location of images i.e.\begin{figure}[H]
% \usepackage{graphicx}  %needed to include png, eps figures
% \usepackage{mwe}
% \usepackage{subfiles}
% \usepackage{url} % correct bad hyphenation here
% \usepackage{xcolor}

% \usepackage[backend=biber,style=ieee]{biblatex} 
% \bibliography{biblio.bib} %your file created using JabRef

\begin{document}

% % paper title
% \title{Reduced Order Models \\ With Moving Domains \\ \normalsize{Literature Review}}

% % author names 
% \author{Enrique Millán Valbuena \\ \normalsize{463 426 8}}% <-this % stops a space
        
% % The report headers
% \markboth{M. Sc. Aerospace Engineering, TU Delft}%do not delete next lines
% {Shell \MakeLowercase{\textit{et al.}}: Bare Demo of IEEEtran.cls for IEEE Journals}

% % make the title area
% \maketitle

% % As a general rule, do not put math, special symbols or citations
% % in the abstract or keywords.
% \begin{abstract}
% TBD
% \end{abstract}

% \begin{IEEEkeywords}
% Reduced basis methods, moving domain, heat equation, Galerkin-projection, FEM, DEIM, MDEIM, POD
% \end{IEEEkeywords}

\section{Introduction}

From  \cite{2016_CertifiedReducedBasisMethodsParametrizedPDE_Hesthaven}:
\begin{quotation}
    The central idea of the reduced basis approach is the identification of a suitable problem-dependent basis to effectively represent parametrized solutions to partial differential equations.
\end{quotation}

\begin{itemize}
    \item Difference between local and global support.
    \item Decomposition techniques.
\end{itemize}

\section{Needs}
We aim at obtaining efficiently the solution of parametrized parabolic PDEs with moving boundaries in time.
Others have solved a problem with a moving mesh in time, but only using DEIM \cite{2018_podDeimReducedOrderModelDeformingMeshAeroelasticApplications_Donfrancesco}.

index-based computational domain in order to deal with deforming grid.

\begin{itemize}
    \item Build a ROM system equivalent to the FEM discretization of the weak form of the PDE.
    \begin{itemize}
        \item Solve the ROM for each time-step and project back to physical domain.
        \item Functional evaluation of the solution.
    \end{itemize}
    \item Do not assemble and project any high fidelity operators to build the ROM operators.
    \begin{itemize}
        \item Sampling strategies across the parameter space are crucial.
            \begin{itemize}
                \item Ensure convergence.
                \item Computational efficiency.
            \end{itemize}
        \item Create a POD-basis for:
        \begin{itemize}
            \item The solution space.
            \item Each operator, matrix or vector, of the system.
            \item The "snapshots method" is used \cite{1987_turbulenceDynamicsCoherentStructures_Sirovich}.
            \item Antonnen et al. \cite{2003_podBasedReducedOrderModelsWithDeformingGrids_anttonen} propose an index-based POD for which the snapshots are collected retaining the index numbering order so that a fixed computational index-based domain is taken into account to preserve the consistency of the POD process.
            \item This approach allows to directly deal with discrete projection and discrete inner product for the construction of the ROM, in contrast with the continuous formulation that requires the use of characteristic functions to follow the moving fluid-structure interface.
        \end{itemize}
        \item Discrete Empirical Interpolation of the ROM operators.
        \item Greedy sampling techniques are similar in objective to, but very different in approach from, the more well-known methods of proper orthogonal decomposition (POD). How are they different?
    \end{itemize}
    \item Certify the ROM solution with a posteriori error bounds.
\end{itemize}

\subsection{Scope}
\begin{itemize}
    \item Prescribed deformation of the domain:
    \begin{itemize}
        \item Interpolate through the mesh with a Laplacian operator for each time-step.
        \item Separable geometrical and time parametrization of the domain deformation.
        \item No FSI problem to be solved.
    \end{itemize}
    \item Linear operators:
    \begin{itemize}
        \item Heat equation.
        \item Convection-diffusion with known velocity field. 
    \end{itemize}
    \item Lifting function for boundary conditions:
    \begin{itemize}
        \item The homogeneous problem is reduced.
        \item The lifting operators are reduced. 
    \end{itemize}
\end{itemize}
\section{What Others Have Done}
\begin{itemize}
    \item POD-Galerkin projection.
    \item Operators reduction:
    \begin{itemize}
        \item (DEIM) Vector reduction and interpolation \cite{2010_nonlinearModelReductionDeim_chaturantabut}.
        \item (MDEIM) Matrix reduction and interpolation \cite{2015_efficientModelReductionParametrizedSystemsMatrixDeim_Negri}.
        \item Reduction of non-linear operators \cite{2018_podDeimReducedOrderModelDeformingMeshAeroelasticApplications_Donfrancesco}.
    \end{itemize}
    \item Problem parametrization: 
    \begin{itemize}
        \item PDE coefficients.
        \item Boundary conditions.
        \item Geometrical deformation of the domain.
    \end{itemize}
\end{itemize}

\section{What We Intend To Do}

\begin{itemize}
    \item Validate with a non-parametrized time deforming mesh.
    \item Reduce a parametrized time-deforming mesh.
\end{itemize}

\subsection{Error Bounds}
Means to certify the construction of the RB model: a posteriori error bounds.

They should be:
\begin{enumerate}
    \item Computable.
    \item Rigorous (i.e. provable).
    \item Effective: they should not arbitrarily overestimate the error.
\end{enumerate}

\subsection{Research Questions}

\begin{itemize}
    \item How does the parameter sampling strategy affect the goodness of the POD-basis?
    \item How does the parameter sampling strategy affect the goodness of the Discrete Empirical Interpolation?
    \item Can we predict the minimum viable number of basis for a given error?
    \item What is a representative snapshot?
    \item When can we consider the basis sufficiently rich?
    \item What does it mean for a basis to be \textit{rich}? 
    \item How does the selection of the inner product change the resultant basis of the POD?
    \item Can we use information from the PDE to improve this inner product?
    \item Why does the POD generate basis functions with global support? 
\end{itemize}

\section{Conclusions}
\label{sec:conclusions}

\newpage
\printbibliography

\end{document}


