\documentclass[thesis.tex]{subfiles}


\begin{document}
\section{Introduction}
When a painter sets out to paint, she will probably use most of the available basic colours.
However, if she knew beforehand she was only going to paint landscapes, 
she would fare well with a farsighted palette: greens, browns, blues, whites, etc.
Such is the nature of Reduced Order Models, 
to find a subset among the combinations of colours to represent the solution to the problem of interest.

In this context, the basic colours are the classical mathematical Lagrangian finite element basis functions, 
generic and with local support to represent most functions of interest.
Instead, the landscape palette will be ad-hoc functions, with global support, good at capturing details only specific to landscapes.

Additionally, she will not need all sorts of brushes, simply the ones with the right thickness and width for mountains and trees.
The brushes represent the algebraic operators that arise from the finite element discretization.
In a similar fashion as with the colours, we can find a subset of combinations of brushes that suit our problem.
That is, we can find a basis for each algebraic operator to build them efficiently.

Finally, since she is a vanguard painter, the domain of our problem, her canvas, will be allowed to change in time, as she paints.
The landscape colours and brushes we select will need to take this into account.

\mytodo{Mention that our goal is to avoid the computation of the jacobian matrix.}
\mytodo{Mention that we will be using ALE formulation.}

% Now that we have briefly stated the goal of our work

% Because we will have a reduced basis for the solution \textit{and} the algebraic operators,
% we say that we have an \textit{hyper}-reduced model, for all of its ingredients have assigned an ad-hoc representation basis.

\end{document}