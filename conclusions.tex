\documentclass[thesis.tex]{subfiles}


\begin{document}

\section{Conclusions}
% The jacobian can be succesfully skipped!
An hyperreduced order model has been succesfully created and certified for
the one-dimensional isentropic moving piston.
This is a simplified gas dynamics problem, 
with all the ingredients that conform a real life problem in their mild versions:
a Burgers-like nonlinear term and a moving boundary.

Our approach to the construction of the Reduced Order Model has been purely algebraic, 
with a non-intrusive approach that should allow its implementation on existing solvers.
We have obtained a reduced basis with global support to carry out the Galerkin projection;
and a collateral basis for each of the algebraic operators (vectors and matrices),
to approximate their projection unto the reduced space 
without explicitly assembling and projecting the original FOM operator.

Hence, it has allowed us to prove the fact that the technique is satisfactory
to solve problems where a jacobian transformation would be required,
without ever actually computing the jacobian. 
However, the technique is actually more powerful, 
as it could be used for domains whose boundary remains fixed,
but whose internal nodes do not.

\subsection{Limitations}
We are aware of how much we have benefited from solving our PDE in a one-dimensional domain.
Despite including nonlinear terms and a moving mesh,
the formulation of the problem remains benign. 

Nevertheless, our whole procedure would scale switfly to higher dimensions.

\subsection{Future Work}
The next natural step would be to extend the methodology to higher-dimensional domains.
Most of the problem would remain the same (albeit larger matrices), 
except for a difficulty which was not present in this work: 
the calculation of the Dirichlet lifting. 

Indeed, in this work we leveraged the advantage that the lifting could be computed analytically.
In higher dimension, the boundary conditions need to be numerically extended to the domain,
either via harmonic extensions
\cite{formaggiaALE},
or solving the elastodynamic equations
\cite{1995_farhat_elasticEquations}.

These are not the only challenges when the problem is expanded to higher dimensions.
For the creation of the Navier-Stokes ROM, the velocity field needs to be enriched
\cite{supremizers},
so that the inf-sup condition is satisfied at the ROM level too.

The oscillating cylinder problem could be a good candidate 
to test the formulation in more realistic setting.

\end{document}