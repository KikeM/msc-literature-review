\documentclass[thesis.tex]{subfiles}


\begin{document}

\section{Conclusions}
An hyperreduced order model has been succesfully created and certified for
the one-dimensional isentropic moving piston.
This is a simplified gas dynamics problem, 
with all the ingredients that conform a real life problem in their mild versions:
a Burgers-like nonlinear term and a moving boundary.

Our approach to the construction of the Reduced Order Model has been
purely algebraic and with a black-box setting.
That is, we have obtained a reduced basis with global support to carry out the Galerkin projection;
and a collateral basis for each of the algebraic operators (vectors and matrices),
to approximate their projection unto the reduced space 
without explicitly assembling and projecting the original FOM operator.

Hence, it has allowed us to prove the fact that the technique is satisfactory
to solve problems where a jacobian transformation would be required,
without ever actually computing the jacobian. 
However, the technique is actually more powerful, 
as it could be used for domains whose boundary remains fixed,
but the internal nodes do not (i.e. shock tracking schemes).

% The jacobian can be succesfully skipped!

\subsection{Future Work}
The next natural step would be to extend the methodology to higher-dimensional domains.
Most of the problem would remain the same (albeit larger matrices), except for a difficulty 
which was not present in this work: the calculation of the Dirichlet lifting. 
Indeed, in this work we leveraged the advantage that the lifting could be computed analytically.
In higher dimension settings, an empirical extension needs to be done,
either via harmonic extensions
\mytodo{Cite!},
or solving the elastic deformation equations
\mytodo{Cite!}.

Use the methodology with the actual Navier-Stokes equations.

The oscillating cylinder problem contains both requirements.

\end{document}