\documentclass[thesis.tex]{subfiles}


\begin{document}
\onecolumn

% \epigraph{
%     \itshape
%     To Federica's smoking corner ... \\
%     (so that one day she doesn't need one anymore).
%     }
    
\epigraph{
    \itshape
    To go to Rome is little profit;
    \\
    to go to Rome is little profit, 
    endless pain. 
    \\
    The master that you seek in Rome, 
    \\
    you find at home,
    or seek in vain.
    }
    
\section*{Foreword}
\addcontentsline{toc}{section}{Foreword}
\vspace{5mm}

This kind of works typically set the ending of a cycle in life.
In my case, it marks a restart.

I was very close to dropping out from my masters, contempt with having my bachelor and partially frustrated for multiple reasons not worth developing here.
Taking into account the fact that my professional life seemed to have drifted away from Aerospace Engineering,
I struggled to see the point at completing it, let alone finding the time such task required.

However, two ideas made me get back to work.
First, it is easier to explain an elongated but completed academic course, rather than an unfinished one.
Second, and most important, I do actually like the subject.
In fact, during the last year I have realized how much I enjoy \textit{Applied Mathematics}, 
a field by which Aerospace Engineering is widely nurtered.

In this regard I would like to thank professors A. Quarteroni and A. Manzoni for giving me the opportunity
to work with them in their research group in Milan. 
Apart from learning mathematics, I met great colleagues and picked up one of Europe's most beautiful languages.
However, working with the FEM code available there, LifeV, was tough, and probably its complexity had to do with the growing frustration I was already experiencing in my academic life:
too much time spent debugging, rather than understanding the problem with pen and paper.
Nevertheless, with them I discovered a way of using mathematics that I had not learnt before, one that suits my mind and approach to scientific modelling.

I am grateful for the warmth I received from the PhD students and postdocs at the office where we worked together everyday: Federica, Dani (south), Dani (north), Abele, Ludovica, Stefano, and a countable infinite more.
I keep good memories playing volley and climbing with you all.
Last, but not least, I would like to thank Niccolò Dal Santo for his patience, time and knowledge; which he generously dedicated to me despite not being officially assigned as my supervisor. 
You are definitely among the most clever and kindest people I have met in my life.

At TU Delft, I would like to thank professor Steven for his joyful encouragement during round two of this thesis.
When I first wrote him back after a year and a half since he last heard from me, I thought he would (righteously) no longer want to have anything to do with this work.
I was gladly surprised to receive all of the contrary, I warm welcome back and a pragmatic view to reach the end at the fastest pace, whilst doing a good job. 

At last, although they are completely outside of the academic scope, I would like to thank my colleagues at work and my close friends.
My two supervisors, Ana and Sarah, from whom I have adquired the pragmatism industrial problems require to be completed in time and form.
To Maximiliano, again a smart and kind person across my path in life, eager to teach me professional coding skills and how to structure creativity.
Emmanuel, my housemate, who despite being a lawyer would kindly ask me every now and then how my convergence rates were going on.
Miquel, a friend turned brother, for your unwearing support and advice.
And to all of the remaining, with whom I spend great quality time.
You influence my life more than you are probably aware.

As the reader will see, this is quite a simple work.
Coding it has been laborious, but the content remains simple.
Yet, its simplicity has allowed me to understand the fundamentals of 
two versatile and powerful mathematical tools: Finite Elements and Reduced Order Models.
This has motivated me to keep on working at it once this is over, 
so that hopefully one day I get to tackle the real-life problem I set to solve in the first place: 
the fluid mechanics problem of the human heart.

\begin{flushright}
    \vspace{5mm}
    \textit{Madrid (España), 2021.}    
\end{flushright}
% This work has taken me longer than I expected to complete. 
% In fact, I am not completely happy with it. 
% I would have rather done a much complex problem, with a 

\end{document}